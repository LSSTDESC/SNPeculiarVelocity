\documentclass{aastex62}   	% use "amsart" instead of "article" for AMSLaTeX format
\usepackage{graphicx}				% Use pdf, png, jpg, or eps§ with pdflatex; use eps in DVI mode
								% TeX will automatically convert eps --> pdf in pdflatex		
\usepackage{amssymb}
\usepackage{natbib}

%SetFonts

%SetFonts

\begin{document}

\title{Testing Gravity using Type Ia Supernovae Discovered by LSST}
\author[0000-0001-6315-8743]{A.~G.~Kim}
\affiliation{    Physics Division, Lawrence Berkeley National Laboratory, 
    1 Cyclotron Road, Berkeley, CA, 94720}
    
\author{and others}

%\date{}							% Activate to display a given date or no date


\maketitle
\section{Introduction}
Peculiar velocities provide a measure of $f\sigma_8$, which in turn probes gravity.  Type~Ia supernovae. as a precise distance indicator,
can in turn  provide precise peculiar velocities of their host galaxies \citep{2006PhRvD..73l3526H,2011ApJ...741...67D}.
\citet{2015JCAP...12..033H, 2017JCAP...05..015H}  test and ultimately find consistency between
$\Lambda$CDM and the peculiar velocities of existing SN~Ia samples,

Surveys such as ZTF and LSST are and will discover orders of magnitude more nearby SNe~Ia than currently available.
The motivation of this work is to quantify the probative power of their peculiar velocities in the LSST era.
While there have been a number of articles on the subject,
our analysis brings a higher level of fidelity than performed in previous analyses.  We simulate SNe~Ia hosted by galaxies in a mock galaxy
catalog. The numbers of SNe are sufficiently small to allow fast evaluations of the likelihood, which enable the determination of parameter
posteriors using MCMC on reasonable timescales.   We can use our machinery to 
compare different survey parameters, such as redshift depth, total numbers of supernovae,
solid angle/survey geometry, and the implications of different SN~ia follow-up strategies.

Galaxy-count -- peculiar velocity cross-correlation is not yet implemented but is a planned extension of this work.  We would like to
quantify the suppression of sample variance by comparing matter-density and velocities in the same volume \citep{2007PhRvL..99h1301G}.

\section{Simulated Data}

\section{Anaysis}

\section{Results}

\bibliographystyle{aasjournal}
\bibliography{/Users/akim/Documents/alex}

\end{document}  