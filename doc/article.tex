\documentclass{aastex62}   	% use "amsart" instead of "article" for AMSLaTeX format
\usepackage{graphicx}				% Use pdf, png, jpg, or eps§ with pdflatex; use eps in DVI mode
								% TeX will automatically convert eps --> pdf in pdflatex		
\usepackage{amssymb}
\usepackage{natbib}

%SetFonts

%SetFonts

\begin{document}

\title{Testing Gravity using Type Ia Supernovae Discovered by LSST}
\author[0000-0001-6315-8743]{A.~G.~Kim}
\affiliation{    Physics Division, Lawrence Berkeley National Laboratory, 
    1 Cyclotron Road, Berkeley, CA, 94720}
\author{C.~Harper}
\affiliation{    Physics Division, Lawrence Berkeley National Laboratory, 
    1 Cyclotron Road, Berkeley, CA, 94720}
\author{C.~Ju}
\affiliation{    Physics Division, Lawrence Berkeley National Laboratory, 
    1 Cyclotron Road, Berkeley, CA, 94720}
\author{D.~Huterer}
\affiliation{Department of Physics, University of Michigan, 450 Church Street, Ann
Arbor, MI 48109, USA }

\author{and others}

%\date{}							% Activate to display a given date or no date


\section{Introduction}
Peculiar velocities provide a measure of $f\sigma_8$, which in turn probes gravity.  As precise distance indicators Type~Ia supernovae
can provide precise peculiar velocities (expressed practically in term of peculiar magnitudes)
of their host galaxies \citep{2006PhRvD..73l3526H,2011ApJ...741...67D}.
\citet{2015JCAP...12..033H, 2017JCAP...05..015H}  test and ultimately find consistency between
$\Lambda$CDM and the peculiar velocities of existing SN~Ia samples.

Surveys such as ZTF and LSST are and will discover orders of magnitude more nearby SNe~Ia than currently available.
The motivation of this work is to quantify the probative power of SN~Ia-derived peculiar velocities in the LSST era.
While there have been a number of articles on the subject,
our analysis brings a higher level of fidelity than sought by previous analyses.  We simulate SNe~Ia hosted by galaxies in a mock galaxy
catalog. The numbers of SNe are sufficiently small to allow fast evaluations of the likelihood, which enable the determination of parameter
posteriors using MCMC on reasonable computing timescales.   We can use our machinery to 
compare different survey parameters, such as redshift depth, total numbers of supernovae,
solid angle/survey geometry, and SN~Ia intrinsic magnitude dispersion.

The cross-correlation between galaxy-count and peculiar velocity surveys is not yet implemented but is a planned extension of this work.  We would like to
quantify the suppression of sample variance achieved when considering matter-densities and velocities 
within the same volume \citep{2007PhRvL..99h1301G}.

\section{Simulated Data}
The Buzzard (v1.6) galaxy catalog is used.  This is because it is the survey, among those currently available
in the Generic Catalog Reader of the DESC, with a light cone that covers the largest solid angle of 10,313.24 sq.~deg.  However,
the survey geometry is different
from that of LSST.  The catalog is based on a Flat $\Lambda$CDM model with $\Omega_M=0.286$, $\Omega_B=0.047$, and
$\Omega_\nu=0$,
The catalog contains140M galaxies with observed $z<0.2$ for which the star formation rate and stellar mass are available; from these  parameters the
supernova rate of each host is determined using the rates of
\citet{2012ApJ...755...61S}.  Supernovae are realized for a span of 10 observer-years yielding 17,205 objects.  This total supernova
production underestimates the expected discovery of 134,574 supernovae in 10 observer-years based on the volumetric
rate of \citet{2010ApJ...713.1026D}.  The volumetric rate is more robust, so the sample under consideration in this article
corresponds to 1.3 years worth of discoveries.

\section{Analysis}
The analysis of this article is identical to that of \citet{2015JCAP...12..033H, 2017JCAP...05..015H}.   The peculiar magnitude correlation
function $\xi_{\delta m \delta m}$ expected from General Relativity, the matter density power spectrum at the CMB assuming the same
cosmological parameters used for the Buzzard catalog, is calculated using CAMB \citep{Lewis:2002ah}.  Our model for the data covariance
is
\begin{equation}
C_{ij} = A\xi_{\delta m \delta m}(\mathbf{r_i},\mathbf{r_j}) + \frac{\sigma_M^2}{N_i} \delta_{ij}.
\end{equation}
In this model, we treat the shape of the power spectrum $\xi$ as fixed, and use the parameter $A$ to represent
deviations from General Relativity $A=1$.  This model is not sensitive to  the shape of
the power spectrum between gravitational models.  Nevertheless, for
convenience we make the association that our model gives $f\sigma_8 = A f\sigma_8^{GR}$, where $ f\sigma_8^{GR}$ is the expectation
from General Relativity.  Special cases of modified gravity could give $A=1$, while inconsistency with $A=1$ is inconsistency with GR.

\section{Results of Subsets}

\bibliographystyle{aasjournal}
\bibliography{/Users/akim/Documents/alex}

\end{document}  