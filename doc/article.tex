\documentclass{aastex62}   	% use "amsart" instead of "article" for AMSLaTeX format
\usepackage{graphicx}				% Use pdf, png, jpg, or eps§ with pdflatex; use eps in DVI mode
								% TeX will automatically convert eps --> pdf in pdflatex		
\usepackage{amssymb}
\usepackage{natbib}
\usepackage{grffile}

%SetFonts

%SetFonts

\begin{document}

\title{Testing Gravity using Type Ia Supernovae Discovered by LSST}
\author[0000-0001-6315-8743]{A.~G.~Kim}
\affiliation{    Physics Division, Lawrence Berkeley National Laboratory, 
    1 Cyclotron Road, Berkeley, CA, 94720}
\author{C.~Harper}
\affiliation{    Physics Division, Lawrence Berkeley National Laboratory, 
    1 Cyclotron Road, Berkeley, CA, 94720}
\author{C.~Ju}
\affiliation{    Physics Division, Lawrence Berkeley National Laboratory, 
    1 Cyclotron Road, Berkeley, CA, 94720}
\author{D.~Huterer}
\affiliation{Department of Physics, University of Michigan, 450 Church Street, Ann
Arbor, MI 48109, USA }

\author{others}

%\date{}							% Activate to display a given date or no date


\section{Introduction}
Peculiar velocities provide a measure of $f\sigma_8$, which in turn probes gravity.  As precise distance indicators Type~Ia supernovae
can provide precise peculiar velocities (expressed equivalently as peculiar magnitudes)
of their host galaxies \citep{2006PhRvD..73l3526H,2011ApJ...741...67D}.
\citet{2015JCAP...12..033H, 2017JCAP...05..015H}  test and ultimately detect correlations in the peculiar velocities of existing SN~Ia samples.

Surveys such as ZTF and LSST are and will discover orders of magnitude more nearby SNe~Ia than currently available.
The motivation of this work is to quantify the probative power of SN~Ia-derived peculiar velocities in the LSST era.
While there have been a number of articles on the subject,
our analysis brings a higher level of fidelity than sought by previous analyses.  We simulate SNe~Ia hosted by galaxies in a mock galaxy
catalog. The numbers of SNe are sufficiently small to allow fast evaluations of the likelihood, which enable the determination of parameter
posteriors using MCMC on reasonable computing timescales.   We can use our machinery to 
compare different survey parameters, such as redshift depth, total numbers of supernovae,
solid angle/survey geometry, and SN~Ia intrinsic magnitude dispersion.

The cross-correlation between galaxy-count and peculiar-velocity surveys is not yet implemented but is a planned extension of this work.  We would like to
quantify the suppression of sample variance achieved when considering matter-densities and velocities 
within the same volume \citep{2007PhRvL..99h1301G}.

\section{Simulated Data}
The Buzzard (v1.6) galaxy catalog is used.  This is because it is the survey, with a light cone that covers 10,313.24 sq.~deg., that covers
the largest solid angle among those currently available
in the DESC Generic Catalog Reader.  However,
the survey geometry is different
from that of LSST.  The catalog is based on a Flat $\Lambda$CDM model with $H_0=70$~km~s$^{-1}$,  $\Omega_M=0.286$, $\Omega_B=0.047$, and
$\Omega_\nu=0$.
The catalog contains 140M galaxies with observed redshift $z<0.2$.
Each galaxy has its cosmological redshift, the $x$-, $y$-, $z$-components of its peculiar velocity from which
the radial peculiar velocity is determined, and its
star formation rate and stellar mass from which the
supernova rate of each host is determined using 
\citet{2012ApJ...755...61S}.  Supernovae are realized for a span of 10 observer-years yielding 17,205 objects.  This total supernova
production underestimates the expected discovery of 130,000 supernovae in 10 observer-years based on the volumetric
rate of \citet{2010ApJ...713.1026D}.  The volumetric rate is more robust, so the sample under consideration in this article
corresponds to 1.3 years worth of discoveries. 

Each supernova is assigned a magnitude based on the distance modulus of its cosmological
redshift plus a random term drawn from a Normal distribution $\sigma_M$, which for simplicity is the same for all supernovae and captures both
intrinsic magnitude dispersion and measurement uncertainty.  No other corrections are applied to the observed magnitude.  Dipole effects of the heliocentric
motion with respect to the CMB and of the galaxies with respect to the CMB  are ignored: referring to
\citet{2011ApJ...741...67D}, the effects in Eq.~18 are ignored.

\section{Analysis}
The analysis of this article is  almost identical to that of \citet{2015JCAP...12..033H, 2017JCAP...05..015H}.   The peculiar magnitude correlation
function $\xi_{\delta m \delta m}$ expected from General Relativity starting with the CMB
matter density power spectrum is calculated using CAMB \citep{Lewis:2002ah}  assuming the same
cosmological parameters used for the Buzzard catalog.  Our model for the data covariance
is
\begin{equation}
C_{ij} = A\xi_{\delta m \delta m}(\mathbf{r_i},\mathbf{r_j}) + \frac{\sigma_M^2}{N_i} \delta_{ij} + \sigma^2_{NL}(z_i;\sigma_{v})\delta_{ij}.
\end{equation}
In this model, we treat the shape of the power spectrum $\xi$ as fixed and use the parameter $A$ to represent
deviations from General Relativity $A=1$.  This model is not sensitive to  the shape of
the power spectrum between gravitational models.  Nevertheless, for
convenience we say that our model gives $f\sigma_8 = A (f\sigma_8)^{GR}$, where $(f\sigma_8)^{GR}$ is the expectation
from General Relativity.  Special cases of modified gravity could give $A=1$, but inconsistency with $A=1$ is is evidence against General Relativity.
The final term includes extra magnitude dispersion produced by non-linear effects on velocity, parameterized with $\sigma_{v}$, such that
\begin{equation}
\sigma_{NL} = \frac{5}{\ln{10}} \frac{1+\bar{z}}{\bar{z}} \sigma_v,
\end{equation}
where $\bar{z}$ is the cosmological redshift of the host galaxy.
\section{Results of Subsets}

For all analyses we remove supernovae below observed $z_{min}=0.01$ in order to
reduce errors made in the first-order transformation between velocity and
magnitude.  This cut removes a  small volume relative to the full survey.

We consider two different scenarios cutting at different redshift depths $z_{max}=0.1$ and  $z_{max}=0.15$.  All the objects in the first
scenario are in the second scenario.  The results are shown in Fig.~\ref{zmax:fig}.  The points of interest are:
\begin{itemize}
\item There is a bias toward low $f\sigma_8$ for low-redshift ($z_{max}=0.1$).  Adding more supernova with increasing redshift
 decreases the bias  ($z_{max}=0.15$), and $A$ eventually approaches 1 ($z_{max}=0.2$ not shown).
 \item Based on the analysis of \citet{2017MNRAS.471.3135H}, I added an intrinsic velocity dispersion $\sigma_v$ (km~s$^{-1}$) due to non-linear effects.  This was not
 in Dragan's model.  My fit finds that this  dispersion is consistent with zero, not well constrained, and not strongly correlated with
 $A$.  The lack of covariance is consistent with Dragan's own tests.
 \item The fit intrinsic magnitude dispersion, on the other hand, is larger than the input $\sigma_M=0.08$ mag.  This  indicates that the catalogs
 contain 
 extra  velocity dispersion better attributed to intrinsic magnitude dispersion rather than correlated peculiar velocities.
\end{itemize}


\begin{figure}
\includegraphics[width=0.5\textwidth]{../out/pvlist.0.08.1234.1.0.1.pkl.2000.png}
\includegraphics[width=0.5\textwidth]{../out/pvlist.0.08.1234.1.0.15.pkl.2000.png}
\caption{Confidence regions for the model parameters.  $A=(f\sigma_8)/(f\sigma_8)_{GR}$, $M$ is the supernova
absolute magnitude, $\sigma_M$ is the magnitude dispersion, and $\sigma_v$ is the non-linear contribution
to peculiar velocity.   The parameters for which I controlled the input, $M=0$ and $\sigma_M=0.08$~mag, are shown in dotted lines.
 Left: A survey truncated at $z_{max}=0.1$.  Right: A survey truncated at $z_{max}=0.15$.
\label{zmax:fig}}
\end{figure}

\begin{itemize}
\item Dragan: Is there some way that the formalism in your article breaks down at low-redshift?  I do remove very low-redshift objects at $z<0.01$,
which I think is conservative.
 \item Risa/Joe: Is there any reasons for the Buzzard catalog to not capture correlated peculiar velocities on small scales (box size to large?) nor capture
 non-linear velocities?  (I don't know how to translate a mass resolution of 2.7e10 $M_{\odot}/h$ into a spatial resolution!)
\end{itemize}

\bibliographystyle{aasjournal}
\bibliography{/Users/akim/Documents/alex}

\end{document}  