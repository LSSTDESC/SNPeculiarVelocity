\documentclass{article}
\usepackage{amsmath}
\usepackage{amssymb}
\usepackage{bm}
\usepackage{physics}
\usepackage{verbatim}
\title{Complete draft}
%\author{Chao Ju, Alex Kim}
\begin{document}
\maketitle


%There is sufficient information in the distribution of peculiar velocities and overdensities of galaxies to test general relativity on the largest scale. Suppose we have a region that encloses $N_{gal}$ number of galaxies in the universe. We can divide the region into $N_{cell}\gg 1$ number of equal volume cells whose size can be made so small that either one or no galaxy is contained in one cell. This is encoded in a vector $\textbf{N}$ of size $N_{cell}$ whose entries are either 0 or 1. For any 2 cells, we can compute their correlation, a function of the distance $r$ between the cells only. In our application, we consider $\xi_{\hat{\delta}\hat{\delta}}$ (the overdensity-overdensity correlation), $\xi_{\hat{\delta} v}$ (overdensity-velocity correlation), $\xi_{v\hat{\delta}}$ (velocity-overdensity correlation), and $\xi_{vv}$ (velocity-velocity correlation).\par

Consider a galaxy count and peculiar velocity survey that covers a volume, which is divided into $N_{cell}$ equal-volume cells.  Each cell will contain some
number of survey galaxies; when the cell volume is large then each cell is likely to contain many galaxies whereas when the cell volume is small each cell is likely
to have either one or no galaxies.  Similarly, when the cell volume is large  its  peculiar velocity is determined combining all the velocity measurements therein,
whereas when the volume is small the cell velocity is likely to be that of the one galaxy therein or there is no measurement, alternatively viewed as an infinite
uncertainty measurement.  We denote $N_{gal}$ as the number of cells that contains at least one galaxy, at the large cell volume limit
$N_{gal}=N_{cell}$ whereas in the limit of infinitely small cell volumes it is
the number of galaxies in the survey.

%The data  is $\textbf{x}= [N_1-\bar{N}, N_2-\bar{N}, \hdots, N_{N_{cell}}-\bar{N}, \hat{\delta} v_1, \hdots, \hat{\delta} v_{N_{gal}}]^T$, where $\hat{\delta} v$ is the deviation of the measured velocity from the expected, 
The data in the cells are denoted as $\textbf{y} = [N_1, N_2, \hdots, N_{N_{cell}}, v_1, \hdots, v_{N_{gal}}]$
and are modeled as $\textbf{y}_0 =[\bar{N}, \bar{N}, \hdots, \bar{N}, v_{0,1}, \hdots, v_{0,N_{gal}}]$.  For now we assume that the model is not fixed but depends
on linear parameters $p_\alpha$ such that
 $\textbf{y}_0 = p_\alpha \textbf{x}_\alpha$.
For example, there could be  a $p_\alpha = H_0$, $x_\alpha=0$ for the counts elements, and $x_\alpha=d_L$ for the velocity elements. 
Their difference is  denoted as $\hat{\delta} \textbf{y}= \textbf{y} - \textbf{y}_0 $ and $\hat{\delta} v= v-v_0$.  (Although the note is written in terms of velocities
in practice we expect to work with magnitudes.)

We model the observations as being drawn from a Gaussian distribution.
The variance in number counts in a cell is $\bar{N}$ and the vector of peculiar velocity measurement in the cells is denoted $\bm{\sigma}_v$,
  Then the above correlation matrices can be used to construct a block matrix $\textbf{C}$:
\begin{align}
\textbf{C} &\equiv \textbf{C}_0 + \textbf{S}\\
%\textbf{C} &\equiv \textbf{C}_0 + A\textbf{S} \\ 
&=
\left(
\begin{array}{cc}
\bar{N}\textbf{I} & 0 \nonumber \\
0 & \bm{\sigma}_v^2\textbf{I}
\end{array}
\right)
+
\begin{pmatrix}
\bar{N}^2\xi_{gg} & \bar{N}\xi_{g v} \\
\bar{N}\xi_{v g} & \xi_{vv}
\end{pmatrix},
\end{align}
where $\xi_{gg}$, $\xi_{vv}$, and $\xi_{g v}$ are the galaxy overdensity and peculiar velocity correlation functions and their cross-correlation
respectively.  
The dimension of $C$ is $N_{cell}+N_{gal}$ by $N_{cell}+N_{gal}$. For the peculiar velocity covariance, we assign $\sigma^2=\infty$ to cells that do not contain any galaxy, and one can show that removing rows and columns that contain those cells has no effect but a constant offset on what we will compute later.


The science parameters $\theta$ enter through the correlation functions, and may include $\theta \in \{fD, bD, \gamma, \Omega_{M_0},\ldots\}$ and others
that describe the matter power spectrum.
For example, the parameter $fD$ enters in through the relations $\xi_{vv}\propto (fD\mu)^2$, the SN~Ia host-galaxy count overdensity
power spectrum $\xi_{g g }\propto (bD + fD\mu^2)^2$, and the galaxy-velocity cross-correlation $\xi_{vg}
\propto  (bD + fD\mu^2)fD$, where $b$ is the galaxy bias and $\mu\equiv \cos{(\hat{k} \cdot \hat{r})}$ where $\hat{r}$ is the direction of
the line of sight.   There are also the parameters $p$ that linearly affect the mean of the Gaussian, a primary example being the combination
$M+5\log{H_0}$ where $M$ is the SN absolute magnitude and $H_0$ is the Hubble constant.


The Gaussian likelihood as a function of $\theta$ is
\[
\mathcal{L}(\theta) = \frac{1}{\sqrt{\left(2\pi\right)^{N_{cell}+N_{gal}} \det \textbf{C}}}\exp\left\{-\frac{1}{2}\hat{\delta} \textbf{y}^T\textbf{C}^{-1}\hat{\delta} \textbf{y}\right\}.
\]
and
\begin{equation}
-2\ln\mathcal{L}(\theta) = \ln\det\textbf{C} + \hat{\delta} \textbf{y}^T\textbf{C}^{-1}\hat{\delta} \textbf{y} + \text{const}.
\label{log:eqn}
\end{equation}


Assuming linear dependence of the parameters,
\begin{align}
\textbf{C} 
&=
\left(
\begin{array}{cc}
\bar{N}\textbf{I} & 0 \nonumber \\
0 & \bm{\sigma}_v^2\textbf{I}
\end{array}
\right)
+ \sum_\alpha \theta_\alpha
\begin{pmatrix}
\bar{N}^2\xi_{gg,{\theta_\alpha}} & \bar{N}\xi_{g v,{\theta_\alpha}} \\
\bar{N}\xi_{v g,{\theta_\alpha}} & \xi_{vv,{\theta_\alpha}}
\end{pmatrix}\\
&=
\textbf{C}_0
+ \sum_\alpha \theta_\alpha \textbf{S}_\alpha,
\end{align} 
where the comma represents the partial derivative.

%\begin{comment}
%The inverse can be expanded as
%\begin{align}
%\textbf{C}^{-1}
%&=
%\textbf{C}_0^{-1} \left( 1 +  \sum_\alpha \theta_\alpha \textbf{S}_\alpha \textbf{C}_0^{-1} \right)^{-1}\\
%&=
%\textbf{C}_0^{-1} \sum_{i=0}^{\infty} (-1)^i \left(\sum_\alpha \theta_\alpha \textbf{S}_\alpha \textbf{C}_0^{-1} \right)^i.
%\end{align} 
%
%In our case
%\begin{align}
% \textbf{S}_\alpha \textbf{C}_0^{-1} & = 
% \begin{pmatrix}
% \bar{N}\xi_{gg,\theta_\alpha} & \frac{\bar{N}}{ \bm{\sigma}_v^2} \xi_{gv,\theta_\alpha}\\
% \xi_{vg,\theta_\alpha} & \frac{1}{ \bm{\sigma}_v^2} \xi_{vv,\theta_\alpha}\\
% \end{pmatrix}
% \end{align}
%
%We will eventually take the limit where $\bar{N} \rightarrow 0$.
%Then
%\begin{align}
%\left(\sum_\alpha \theta_\alpha \textbf{S}_\alpha \textbf{C}_0^{-1} \right)^i &  \approx 
%\sum_{\alpha}  \ldots \sum_{\hat{\delta}}  \theta_{\alpha} \ldots \theta_\hat{\delta}
%  \begin{pmatrix}
% \frac{\bar{N}}{ \bm{\sigma}_v^{2(i-1)}} \xi_{gv,\theta_\alpha}  \xi_{vg,\theta_\beta} \ldots  \xi_{vg,\theta_\gamma} & \frac{\bar{N}}{ \bm{\sigma}_v^{2i}} \xi_{gv,\theta_\alpha} \xi_{vv,\theta_\beta} \ldots  \xi_{vv,\theta_\gamma} \\
% \frac{1}{ \bm{\sigma}_v^{2(i-1)}} \xi_{vv,\theta_\alpha} \xi_{vg,\theta_\beta}\ldots  \xi_{vg,\theta_\gamma}   & \frac{1}{ \bm{\sigma}_v^{2i}} \xi_{vv,\theta_\alpha} \xi_{vv,\theta_\beta}\ldots  \xi_{vv,\theta_\gamma}\\
% \end{pmatrix}.
%\end{align}
%\end{comment}
%\left(
%\begin{array}{cc}
%\bar{N}\textbf{I} & 0 \nonumber \\
%0 & \bm{\sigma}_v^2\textbf{I}
%\end{array}
%\right)
%+
%A
%\left(
%\begin{array}{cc}
%\bar{N}^2\xi_{\hat{\delta}\hat{\delta}} & \bar{N}\xi_{\hat{\delta} v} \\
%\bar{N}\xi_{\hat{\delta} v} & \xi_{vv}
%\end{array}
%\right).
%\end{align}\par



The estimator and Fisher matrix for this Gaussian likelihood are well known (and derived in  \S\ref{appendix:sec}).
\begin{align}
\hat{\theta}_\alpha & = \frac{1}{2} F^{(\theta \theta)\,-1}_{\alpha \beta} [(\hat{\delta} \textbf{y}^T \textbf{C}^{-1} \textbf{S}_\beta \textbf{C}^{-1} \hat{\delta} \textbf{y}) - \tr(\textbf{C}^{-1} \textbf{S}_\beta \textbf{C}^{-1} \textbf{C}_0)] \nonumber \\
\hat{p}_\alpha  &=  F^{(pp)\, -1}_{\alpha \beta}  (\textbf{y}^T  \textbf{C}^{-1} x_\beta)
\label{estimators:eqn}
\end{align}
where  
\begin{align}
F^{(\theta\theta)}_{\alpha\beta}& = \frac{1}{2} \tr(\textbf{C}^{-1} \textbf{S}_\alpha \textbf{C}^{-1}  \textbf{S}_\beta)\\
F^{(pp)}_{\alpha\beta} & =x^T_\alpha \textbf{C}^{-1} x_\beta\\
F^{(\theta p)}_{\alpha\beta} & = 0.
\end{align}

%Let us consider the limit where there are an infinite number of cells and $\bar{N} \rightarrow 0$ and $\bar{N}\sum_{i,j\in N_{cell}} = n \int dV$, where $n$ is the galaxy number density,
Denoting $\textbf{M}_\beta = \textbf{C}^{-1} \textbf{S}_\beta \textbf{C}^{-1}$ and splitting the calculation into quadrants  corresponding to the galaxy and velocity indices,
the complicated terms are
\begin{align*}
\hat{\delta} \textbf{y}^T \textbf{C}^{-1} \textbf{S}_\beta \textbf{C}^{-1} \hat{\delta} \textbf{y}  
 %& = \frac{1}{\bar{N}^2}\int \int n(x_1) n(x_2) (N(x_1)-\bar{N}) M_{gg}(x_1,x_2) (N(x_2)-\bar{N})  d^3x_1  d^3x_2\\
%& +   \frac{2}{\bar{N}}\int \sum_{i=1}^{N} n(x_1) (N(x_1)-\bar{N}) M_{gv}(x_1,x_i) \hat{\delta} v_i  d^3x_1 \\
%& + \sum_{i=1}^{N}  \sum_{j=1}^{N}  \hat{\delta} v_i M_{vv}(x_i,x_j) \hat{\delta} v_j\\
&= \sum_{i=1}^{N_{\text gal}}  \sum_{j=1}^{N_{\text gal}} N_i M_{\beta,gg}(x_i,x_j) N_j 
 - 2\bar{N}   \sum_{i=1}^{N_{\text cell}} \sum_{j=1}^{N_{\text gal}}  M_{\beta,gg}(x_i,x_j) N_j \\
& +\bar{N}^2 \sum_{i=1}^{N_{\text cell}} \sum_{j=1}^{N_{\text cell}}  M_{\beta,gg}(x_i,x_j) 
 +   2 \sum_{i=1}^{N_{\text gal}}  \sum_{j=1}^{N_{\text gal}}  N_i M_{\beta,gv}(x_i,x_j) \hat{\delta} v_j  \\
& -   2 \bar{N}  \sum_{i=1}^{N_{\text cell}} \sum_{j=1}^{N_{\text gal}}  M_{\beta,gv}(x_i,x_j) \hat{\delta} v_j 
+ \sum_{i=1}^{N_{\text gal}}  \sum_{j=1}^{N_{\text gal}}  \hat{\delta} v_i M_{\beta,vv}(x_i,x_j) \hat{\delta} v_j ,
\end{align*}
\begin{align*}
\tr (\textbf{C}^{-1} \textbf{S}_\beta \textbf{C}^{-1} \textbf{S}_\alpha) & 
=  \sum_{i=1}^{N_{\text cell}} \sum_{j=1}^{N_{\text cell}} M_{\beta,gg}(x_i,x_j)\textbf{S}_{\alpha,gg}(x_j,x_i)  +  2 \sum_{i=1}^{N_{\text cell}} \sum_{j=1}^{N_{\text gal}} M_{\beta,gv}(x_i,x_j)\textbf{S}_{\alpha,vg} (x_j,x_i)\\
& + \sum_{i=1}^{N_{\text gal}} \sum_{j=1}^{N_{\text gal}} M_{\beta,vv}(x_i,x_j) \textbf{S}_{\alpha,vv}(x_j,x_i),
\end{align*}
\begin{align*}
\tr (\textbf{C}^{-1} \textbf{S}_\beta \textbf{C}^{-1} \textbf{C}_0) & 
=  \sum_{i=1}^{N_{\text cell}} \sum_{j=1}^{N_{\text cell}} M_{\beta,gg}(x_i,x_j)\textbf{C}_{0,gg}(x_j,x_i)  + \sum_{i=1}^{N_{\text gal}} \sum_{j=1}^{N_{\text gal}} M_{\beta,vv}(x_i,x_j) \textbf{C}_{0,vv}(x_j,x_i)\\
 & 
= \bar{N} \sum_{i=1}^{N_{\text cell}} M_{\beta,gg}(x_i,x_i)  + \sum_{i=1}^{N_{\text gal}}  M_{\beta,vv}(x_i,x_i) \sigma^{2}_{vi} .
\end{align*}

An interesting limit is when there are infinitely small cell sizes.  Then $N_{gal}$ (the number of cells with galaxies) approaches the number
of galaxies, those cells have $N_i=1$,  and $\bar{N}\sum_{i\in N_{cell}} = n \int dV$, where $n$ is the galaxy number density.
The calculation then is then `binless', with sums over sample objects and integrals to account for the volume where there are no objects.
In this limit, a Poisson rather than Gaussian model distribution better describes expectations. 

Finding the roots of Eq.~\ref{estimators:eqn} to get the estimators is non-trivial.  In particular the $\hat{\theta}$ appears in $\mathbf{C}^{-1}$, a large matrix that
isn't obviously easy to invert.  Using the renormalization group, \cite{2019PhRvD..99d3538M} reparameterizes the likelihood such that much of the parameter-dependence
gets shifted to other terms leaving a smaller
band-diagonal matrix that remains to be inverted.  Applying his methodology to our problem is an interesting direction to pursue.

For our purposes, having a precise answer is probably unnecessary.
Solving the right hand side of Eq.~\ref{estimators:eqn} to lowest (zeroth) order in $\theta$ gives a useful approximate answer.  Replacing
$\textbf{C}^{-1}$ with the diagonal $\textbf{C}_0^{-1}$ gives
\begin{align*}
\textbf{M}_{\beta,gg} & \approx \xi_{gg,\theta_\beta}\\
\textbf{M}_{\beta,gv} & \approx \xi_{gv,\theta_\beta} \bm{\sigma}^{-2}_v\\
\textbf{M}_{\beta,vv} & \approx \bm{\sigma}^{-2}_v \xi_{vv,\theta_\beta} \bm{\sigma}^{-2}_v .
\end{align*}
so that
\begin{align*}
\hat{\delta} \textbf{y}^T \textbf{C}^{-1} \textbf{S}_\beta \textbf{C}^{-1} \hat{\delta} \textbf{y}  
&\approx \sum_{i=1}^{N_{\text gal}}  \sum_{j=1}^{N_{\text gal}} N_i \xi_{gg,\theta_\beta}(x_i,x_j) N_j 
 - 2\bar{N}   \sum_{i=1}^{N_{\text cell}} \sum_{j=1}^{N_{\text gal}}  \xi_{gg,\theta_\beta}(x_i,x_j)N_j \\
& +\bar{N}^2 \sum_{i=1}^{N_{\text cell}} \sum_{j=1}^{N_{\text cell}} \xi_{gg,\theta_\beta}(x_i,x_j)
 +   2 \sum_{i=1}^{N_{\text gal}}  \sum_{j=1}^{N_{\text gal}}  N_i  \xi_{gv,\theta_\beta}(x_i,x_j)\frac{\hat{\delta} v_j}{\sigma^2_{v,j}} \\
& -   2 \bar{N}  \sum_{i=1}^{N_{\text cell}} \sum_{i=1}^{N_{\text gal}}   \xi_{gv,\theta_\beta}(x_i,x_j) \frac{\hat{\delta} v_j}{\sigma^2_{v,j}} 
+ \sum_{i=1}^{N_{\text gal}}  \sum_{j=1}^{N_{\text gal}}   \frac{\hat{\delta} v_i}{\sigma^2_{v,i}}  \xi_{vv,\theta_\beta}(x_i,x_j)  \frac{\hat{\delta} v_j}{\sigma^2_{v,j}},\\
%\hat{\delta} \textbf{y}^T \textbf{C}^{-1} \textbf{S}_\beta \textbf{C}^{-1} \hat{\delta} \textbf{y} & \approx \sum_{i,j\in N_{gal}} N_i N_j \xi_{\hat{\delta}\hat{\delta},\theta_\beta}(x_i,x_j) -2 \bar{N} \sum_{i\in N_{gal},j\in N_{cell}} \xi_{\hat{\delta}\hat{\delta},\theta_\beta,ij}\\
%& +\bar{N}^2  \sum_{i,j\in N_{cell}}  \xi_{\hat{\delta}\hat{\delta},\theta_\beta,ij} + 2 \sum_{i,j\in N_{gal}} \frac{\hat{\delta} v_j}{\sigma_{v,j}^2} \xi_{\hat{\delta} v,\theta_\beta,ij} \\
%& -2\bar{N}\sum_{i\in N_{cell}, j\in N_{gal}} \frac{\hat{\delta} v_j}{\sigma_{v,j}^2} \xi_{\hat{\delta} v,\theta_\beta,ij}  + \sum_{i,j\in N_{gal}} \frac{\hat{\delta} v_i}{\sigma_{v,i}^2} \xi_{vv,\theta_\beta,ij} \frac{\hat{\delta} v_j}{\sigma_{v,j}^2}\\
%& = \sum_{i,j\in N_{gal}} \left( \xi_{\hat{\delta}\hat{\delta},\theta_\beta}(x_i,x_j) + 2  \frac{\hat{\delta} v_j}{\sigma_{v,j}^2} \xi_{\hat{\delta} v,\theta_\beta}(x_i,x_j) + \frac{\hat{\delta} v_i}{\sigma_{v,i}^2} \xi_{vv,\theta_\beta}(x_i,x_j) \frac{\hat{\delta} v_j}{\sigma_{v,j}^2}\right)\\
%&-2  \sum_{i\in N_{gal}}\int n(x_i) \left(\xi_{\hat{\delta}\hat{\delta},\theta_\beta}(x_i,y) + \frac{\hat{\delta} v_i}{\sigma_{v,i}^2} \xi_{v \hat{\delta},\theta_\beta}(x_i,y) \right) d^3y\\
%&  + \int n(x) n(y) \xi_{\hat{\delta}\hat{\delta},\theta_\beta}(x,y)d^3xd^3y.
\end{align*}
\begin{align*}
\tr (\textbf{C}^{-1} \textbf{S}_\beta \textbf{C}^{-1} \textbf{S}_\alpha) & 
= \bar{N}^2  \sum_{i=1}^{N_{\text cell}} \sum_{j=1}^{N_{\text cell}} \xi_{gg,\theta_\beta}(x_i,x_j)\xi_{gg,\theta_\alpha}(x_j,x_i) 
+2 \bar{N}   \sum_{i=1}^{N_{\text cell}} \sum_{j=1}^{N_{\text gal}}\frac{1}{\sigma^{2}_{vj}}   \xi_{gv,\theta_\beta} (x_i,x_j)\xi_{\alpha,vg} (x_j,x_i)\\
& + \sum_{i=1}^{N_{\text gal}} \sum_{j=1}^{N_{\text gal}} \frac{1}{\sigma^{2}_{vi} \sigma^{2}_{vj}} \xi_{vv,\theta_\beta}(x_i,x_j)\xi_{vv,\theta_\alpha}(x_j,x_i) 
\end{align*}
\begin{align*}
\tr (\textbf{C}^{-1} \textbf{S}_\beta \textbf{C}^{-1} \textbf{C}_0) & 
=   \bar{N} \sum_{i=1}^{N_{\text cell}}  \xi_{gg,\theta_\beta}(x_i,x_i)  + \sum_{i=1}^{N_{\text gal}} \frac{1}{ \sigma^{2}_{vi}}\xi_{vv,\theta_\alpha}(x_i,x_i)
\end{align*}

The SN~Ia rate is $2.69 \times 10^{-5} \text{Mpc}^{-3} \text{yr}^{-1}$, meaning that after 10 years there
will be around 1 supernova every 3700~$\text{Mpc}^{3}$.
The Universe is $2.4 \times 10^9$~Mpc$^3$ out to $z=0.2$.  A rough bound on the number of necessary cells is $6\times 10^{5}$.
When taking the linear approximation the calculation scales as $N_{cell}^2$, so this is possible to do on a laptop.
The exact solution requires matrix factorizations that scale as $N_{cell}^3$, which is not so possible on a laptop.

%%\begin{comment}
%%We focus on the parameters $\theta$ related to the covariance matrix.
%%The term
%%\begin{align}
%%\hat{\delta} \textbf{y}^T \textbf{C}^{-1} \textbf{S}_\beta \textbf{C}^{-1} \hat{\delta} \textbf{y} & \approx \hat{\delta} \textbf{y}^T \textbf{C}_0^{-1} \textbf{S}_\beta \textbf{C}_0^{-1} \hat{\delta} \textbf{y}\\
%%&=(\textbf{N}-\bar{N}) \xi_{\hat{\delta}\hat{\delta},\theta_\beta}  (\textbf{N}-\bar{N})  + 2  (\textbf{N}-\bar{N}) \xi_{\hat{\delta} v,\theta_\beta}\frac{\mathbf{\hat{\delta} v}}{\bm{\sigma}_v^2}+ 
%%\frac{\mathbf{\hat{\delta} v}}{\bm{\sigma}_v^2} \xi_{vv,\theta_\beta}\frac{\mathbf{\hat{\delta} v}}{\bm{\sigma}_v^2}.
%%\end{align}
%%Noting that $N_i=1$ when the cell $i$ contains a galaxy and zero otherwise and that only cells with a peculiar velocity measurement contribute ($\sigma_v =\infty$
%%for cells without such a measurement), and that in the limit of infinitely small cells, $\bar{N}\sum_{i\in N_{cell}} = \int n dV$, where $n$ is the galaxy number density,
%%\end{comment}
%
%
%
%The trace term depends on only the zero-lag correlation
%\begin{align}
% \tr(\textbf{C}^{-1} \textbf{S}_\beta \textbf{C}^{-1} \textbf{C}_0) & \approx  \tr(\textbf{C}_0^{-1} \textbf{S}_\beta) \\
% & = \bar{N} \tr(\xi_{\hat{\delta}\hat{\delta},\beta}) + \tr(\frac{1}{\sigma^2_v} \xi_{vv,\beta})\\
%% & = \int n(x) \xi_{\hat{\delta} \hat{\delta},\beta}(x,x) d^3x +  \sum_{i\in N_{gal}} \frac{1}{\sigma^2_{v,i}} \xi_{vv,\beta}(x_i,x_i).
%\end{align}
%
%The term in the Fisher matrix, to the leading term in $\bar{N}$ is
%\begin{align}
% \tr(\textbf{C}^{-1} \textbf{S}_\alpha \textbf{C}^{-1} \textbf{S}_\beta) & \approx  \tr(\textbf{C}_0^{-1} \textbf{S}_\alpha \textbf{C}_0^{-1} \textbf{S}_\beta)  \\
% & = \bar{N}^2 \tr(\xi_{\hat{\delta} \hat{\delta},\alpha}\xi_{\hat{\delta} \hat{\delta},\beta})+
% \bar{N} \tr(\xi_{\hat{\delta} v,\alpha} \frac{1}{\bm{\sigma}^2_v} \xi_{v\hat{\delta},\beta} ) + \tr(\frac{1}{\bm{\sigma}^2_v} \xi_{vv,\alpha}  \frac{1}{\bm{\sigma}^2_v}  \xi_{vv,\beta})\\
% & = \int n(x) n(y) \xi_{\hat{\delta} \hat{\delta},\alpha}(x,y)\xi_{\hat{\delta} \hat{\delta},\beta}(y,x) d^3x d^3y \\
% & + \int n(x) \sum_{i\in N_{gal}} \xi_{\hat{\delta} v,\alpha}(x,x_i) \frac{1}{\sigma^2_{v,i}} \xi_{v\hat{\delta},\beta} (x_i,x) d^3x \\
% & +  \sum_{i,j\in N_{gal}} \frac{1}{\sigma^2_{v,i}} \xi_{vv,\alpha}(x_i,x_j)  \frac{1}{\sigma^2_{v,j}}  \xi_{vv,\beta}(x_j,x_i).
%\end{align}


%  ---- Change
%NEW STUFF
\begin{comment}
We now expand the log-likelihood as a Taylor series for small $\theta$ around $\theta=0$.

First, we focus on the  term $\hat{\delta} \textbf{y}^T\textbf{C}^{-1}\hat{\delta} \textbf{y}$. We rewrite $\textbf{C}$ by normalizing out its uncorrelated $\textbf{C}_0$ term
by   introducing the matrices $\textbf{M}$ and $\hat{\delta} \textbf{y}$,
where $\textbf{M}=\sqrt{\textbf{C}_0}$ and $\textbf{S} = \textbf{M} \hat{\delta} \textbf{y} \textbf{M}$ such that $\hat{\delta} \textbf{y}=\textbf{M}^{-1}\textbf{S}\textbf{M}^{-1}$. Now, our matrix $\textbf{C}$ can be re-written in terms of $\hat{\delta} \textbf{y}$ and $\textbf{M}$:
\begin{equation}
\textbf{C} = \textbf{M}(\textbf{I}+A\hat{\delta} \textbf{y})\textbf{M}.
\end{equation}
Expanded as a Taylor series
\begin{align}
\textbf{C}^{-1} & = \textbf{M}^{-1}(1-A\hat{\delta} \textbf{y}+A^2 \hat{\delta} \textbf{y}^2 + \ldots) \textbf{M}^{-1} \\
& = \textbf{C}_0^{-1}   -A \textbf{C}_0^{-1}\textbf{S}\textbf{C}_0^{-1} +\frac{A^2}{2} \textbf{C}_0^{-1}\textbf{S}\textbf{C}_0^{-1}\textbf{S}\textbf{C}_0^{-1} +O(A^3) + \ldots 
\end{align}
%= -A \textbf{C}_0^{-1}\textbf{S}\textbf{C}_0^{-1} + \text{const}.
%\] \par


A term of interest is 
\begin{align}
\hat{\delta} \textbf{y}^T \textbf{C}_0^{-1}\textbf{S}\textbf{C}_0^{-1} \hat{\delta} \textbf{y}& = (\textbf{N}-\bar{N}) \xi_{\hat{\delta}\hat{\delta}}  (\textbf{N}-\bar{N})  + 2  (\textbf{N}-\bar{N}) \xi_{\hat{\delta} v}\frac{\mathbf{\hat{\delta} v}}{\bm{\sigma}_v^2}+ 
\frac{\mathbf{\hat{\delta} v}}{\bm{\sigma}_v^2} \xi_{vv}\frac{\mathbf{\hat{\delta} v}}{\bm{\sigma}_v^2}.
\end{align}
Noting that $N_i=1$ when the cell $i$ contains a galaxy and zero otherwise and that only cells with a peculiar velocity measurement contribute ($\sigma_v =\infty$
for cells without such a measurement), and that in the limit of infinitely small cells, $\bar{N}\sum_{i,j\in N_{cell}} = n \int dV$, where $n$ is the galaxy number density,
\begin{align*}
\hat{\delta} \textbf{y}^T \textbf{C}_0^{-1}\textbf{S}\textbf{C}_0^{-1} \hat{\delta} \textbf{y}& = \sum_{i,j\in N_{gal}} \xi_{\hat{\delta}\hat{\delta},ij} -2 \bar{N} \sum_{i\in N_{gal},j\in N_{cell}} \xi_{\hat{\delta}\hat{\delta},ij}\\
& +\bar{N}^2  \sum_{i,j\in N_{cell}}  \xi_{\hat{\delta}\hat{\delta},ij} + 2 \sum_{i,j\in N_{gal}} \frac{\hat{\delta} v_j}{\sigma_{v,j}^2} \xi_{\hat{\delta} v,ij} \\
& -2\bar{N}\sum_{i\in N_{cell}, j\in N_{gal}} \frac{\hat{\delta} v_j}{\sigma_{v,j}^2} \xi_{\hat{\delta} v,ij}  + \sum_{i,j\in N_{gal}} \frac{\hat{\delta} v_i}{\sigma_{v,i}^2} \xi_{vv,ij} \frac{\hat{\delta} v_j}{\sigma_{v,j}^2}\\
& = \sum_{i,j\in N_{gal}} \left( \xi_{\hat{\delta}\hat{\delta},ij} + 2  \frac{\hat{\delta} v_j}{\sigma_{v,j}^2} \xi_{\hat{\delta} v,ij} + \frac{\hat{\delta} v_i}{\sigma_{v,i}^2} \xi_{vv,ij} \frac{\hat{\delta} v_j}{\sigma_{v,j}^2}\right)\\
&-2 n \sum_{i\in N_{gal}}\int \left(\xi_{\hat{\delta}\hat{\delta},iV} + \frac{\hat{\delta} v_i}{\sigma_{v,i}^2} \xi_{v \hat{\delta},iV}\right) dV\\
& +n^2  \int \xi_{\hat{\delta}\hat{\delta},V_1V_2}dV_1dV_2.
\end{align*}


We evaluate $\hat{\delta} \textbf{y}^T \textbf{C}^{-1}\hat{\delta} \textbf{y}$ by splitting up the counts and velocities.  The data vector $\hat{\delta} \textbf{y}$ is composed of the counts
 $\textbf{N}-\bar{N}$ and peculiar velocities $\mathbf{\hat{\delta} v}$, whereas $\textbf{S}$ is comprised of $\bar{N}^2\xi_{\hat{\delta}\hat{\delta}}$, $\bar{N}\xi_{\hat{\delta} v}$,
and $\xi_{vv}$.  In this manner
\begin{align}
\hat{\delta} \textbf{y}^T \textbf{C}_0^{-1}\textbf{S}\textbf{C}_0^{-1} \hat{\delta} \textbf{y}& = (\textbf{N}-\bar{N}) \xi_{\hat{\delta}\hat{\delta}}  (\textbf{N}-\bar{N})  + 2  (\textbf{N}-\bar{N}) \xi_{\hat{\delta} v}\frac{\mathbf{\hat{\delta} v}}{\bm{\sigma}_v^2}+ 
\frac{\mathbf{\hat{\delta} v}}{\bm{\sigma}_v^2} \xi_{vv}\frac{\mathbf{\hat{\delta} v}}{\bm{\sigma}_v^2}.
\end{align}
Noting that $N_i=1$ when the cell $i$ contains a galaxy and zero otherwise and that only cells with a peculiar velocity measurement contribute ($\sigma_v =\infty$
for cells without such a measurement), and that in the limit of infinitely small cells, $\bar{N}\sum_{i,j\in N_{cell}} = n \int dV$, where $n$ is the galaxy number density,
\begin{align*}
\hat{\delta} \textbf{y}^T \textbf{C}_0^{-1}\textbf{S}\textbf{C}_0^{-1} \hat{\delta} \textbf{y}& = \sum_{i,j\in N_{gal}} \xi_{\hat{\delta}\hat{\delta},ij} -2 \bar{N} \sum_{i\in N_{gal},j\in N_{cell}} \xi_{\hat{\delta}\hat{\delta},ij}\\
& +\bar{N}^2  \sum_{i,j\in N_{cell}}  \xi_{\hat{\delta}\hat{\delta},ij} + 2 \sum_{i,j\in N_{gal}} \frac{\hat{\delta} v_j}{\sigma_{v,j}^2} \xi_{\hat{\delta} v,ij} \\
& -2\bar{N}\sum_{i\in N_{cell}, j\in N_{gal}} \frac{\hat{\delta} v_j}{\sigma_{v,j}^2} \xi_{\hat{\delta} v,ij}  + \sum_{i,j\in N_{gal}} \frac{\hat{\delta} v_i}{\sigma_{v,i}^2} \xi_{vv,ij} \frac{\hat{\delta} v_j}{\sigma_{v,j}^2}\\
& = \sum_{i,j\in N_{gal}} \left( \xi_{\hat{\delta}\hat{\delta},ij} + 2  \frac{\hat{\delta} v_j}{\sigma_{v,j}^2} \xi_{\hat{\delta} v,ij} + \frac{\hat{\delta} v_i}{\sigma_{v,i}^2} \xi_{vv,ij} \frac{\hat{\delta} v_j}{\sigma_{v,j}^2}\right)\\
&-2 n \sum_{i\in N_{gal}}\int \left(\xi_{\hat{\delta}\hat{\delta},iV} + \frac{\hat{\delta} v_i}{\sigma_{v,i}^2} \xi_{v \hat{\delta},iV}\right) dV\\
& +n^2  \int \xi_{\hat{\delta}\hat{\delta},V_1V_2}dV_1dV_2.
\end{align*}

Now consider the next term $\textbf{C}_0^{-1}\textbf{S}\textbf{C}_0^{-1}\textbf{S}\textbf{C}_0^{-1}$.  The middle section is
\begin{align}
\textbf{S}\textbf{C}_0^{-1}\textbf{S} & =
\begin{pmatrix}
 \bar{N}^2  \xi_{\hat{\delta} v}\frac{1}{\bm{\sigma}_v^2}  \xi_{v \hat{\delta}}  & \bar{N} \xi_{\hat{\delta} v} \frac{1} {\bm{\sigma}_v^2} \xi_{vv}  \\
 \bar{N} \xi_{v v} \frac{1} {\bm{\sigma}_v^2} \xi_{v \hat{\delta}}   &  \xi_{vv} \frac{1}{\bm{\sigma}_v^2} \xi_{vv}
\end{pmatrix}
\end{align}
to leading order of $\bar{N}$ anticipating that we will take the limit where it goes to zero.
Then
\begin{align}
\hat{\delta} \textbf{y}^T \textbf{C}_0^{-1}\textbf{S}\textbf{C}_0^{-1}\textbf{S}\textbf{C}_0^{-1} \hat{\delta} \textbf{y}
& = \sum_{i,j\in N_{gal}} \left( \left(\xi_{\hat{\delta} v}\frac{1}{\bm{\sigma}_v^2}  \xi_{v \hat{\delta}}\right)_{ij}
 + 2  \frac{\hat{\delta} v_j}{\sigma_{v,j}^2} \left(\xi_{\hat{\delta} v} \frac{1} {\bm{\sigma}_v^2} \xi_{vv}  \right)_{ij} + \frac{\hat{\delta} v_i}{\sigma_{v,i}^2} \left( \xi_{vv} \frac{1}{\bm{\sigma}_v^2} \xi_{vv}\right)_{ij} \frac{\hat{\delta} v_j}{\sigma_{v,j}^2}\right)\\
&-2 n \sum_{i\in N_{gal}}\int \left( \left(\xi_{\hat{\delta} v}\frac{1}{\bm{\sigma}_v^2}  \xi_{v \hat{\delta}}\right)_{iV} + \frac{\hat{\delta} v_i}{\sigma_{v,i}^2} \left( \xi_{v v} \frac{1} {\bm{\sigma}_v^2} \xi_{v \hat{\delta}}  \right)_{iV} \right) dV\\
& +n^2  \int  \left(\xi_{\hat{\delta} v}\frac{1}{\bm{\sigma}_v^2}  \xi_{v \hat{\delta}}\right)_{V_1V_2}dV_1dV_2.
\end{align}

We have assembled the contributions of $\hat{\delta} \textbf{y}^T \textbf{C}^{-1}\hat{\delta} \textbf{y}$ to second order in $A$.

Returning to the other contribution to the log-likelihood in Eq.~\ref{log:eqn} we have
\begin{align*}
\ln\det(\textbf{C}) &=\ln\det(\textbf{M}(\textbf{I}+A\hat{\delta} \textbf{y})\textbf{M})) \\
&= \ln\det(\textbf{I}+A\hat{\delta} \textbf{y}) +\text{const} \\
&= A \tr \hat{\delta} \textbf{y} -\frac{A^2}{2}\tr \hat{\delta} \textbf{y}^2+ O(A^3) + \ldots\\
\end{align*}
Now
\begin{equation}
\hat{\delta} \textbf{y} = \textbf{M}^{-1}\textbf{S}\textbf{M}^{-1} = 
\begin{pmatrix}
\bar{N}\xi_{\hat{\delta}\hat{\delta}} & \frac{\sqrt{\bar{N}}}{\sigma_{v}} \xi_{\hat{\delta} v} \\
\frac{\sqrt{\bar{N}}}{\sigma_{v}} \xi_{v\hat{\delta}} & \frac{1}{\sigma_{v}^2} \xi_{vv} 
\end{pmatrix}.
\end{equation}
The contribution to the trace of the upper diagonal in the limit of small cell volumes is
\begin{equation}
\bar{N} \tr \xi_{\hat{\delta}\hat{\delta}} = n \int \xi_{\hat{\delta}\hat{\delta},VV} dV
\end{equation}
and the lower diagonal is
\begin{equation}
\tr  \frac{1}{\sigma_{v}^2} \xi_{vv} =  \sum_{i \in N_{gal}}   \frac{1}{\sigma_{v,i}^2} \xi_{vv,ii}.
\end{equation}

The trace of $\hat{\delta} \textbf{y}^2$ has terms, to leading order in $\bar{N}$, has terms
\begin{equation}
\bar{N}  \tr  \xi_{\hat{\delta} v} \frac{1}{\sigma_{v}}  \frac{1}{\sigma_{v}}  \xi_{v \hat{\delta}} = n \int \left( \xi_{\hat{\delta} v} \frac{1}{\sigma_{v}}  \frac{1}{\sigma_{v}}  \xi_{v \hat{\delta}} \right)_{VV} dV
\end{equation}
and 
\begin{equation}
\tr   \xi_{vv} \frac{1}{\sigma_{v}^2}  \frac{1}{\sigma_{v}^2} \xi_{vv} =  \sum_{i \in N_{gal}} \left(  \xi_{vv} \frac{1}{\sigma_{v}^2}  \frac{1}{\sigma_{v}^2} \xi_{vv} \right)_{ii}
\end{equation}

Putting everything together, the expansion of the log-likelihood in $A$ has the first order coefficient
\begin{align}
c_1 &=- \sum_{i,j\in N_{gal}} \left( \xi_{\hat{\delta}\hat{\delta},ij} + 2  \frac{\hat{\delta} v_j}{\sigma_{v,j}^2} \xi_{\hat{\delta} v,ij} + \frac{\hat{\delta} v_i}{\sigma_{v,i}^2} \xi_{vv,ij} \frac{\hat{\delta} v_j}{\sigma_{v,j}^2}\right)\\
& +\sum_{i \in N_{gal}} \left( \frac{1}{\sigma_{v,i}^2} \xi_{vv,ii}   + 2n \int \left(\xi_{\hat{\delta}\hat{\delta},iV} + \frac{\hat{\delta} v_i}{\sigma_{v,i}^2} \xi_{v \hat{\delta},iV}\right) dV  \right)\\
& +n \int \xi_{\hat{\delta}\hat{\delta},VV} dV  -n^2  \int \xi_{\hat{\delta}\hat{\delta},V_1V_2}dV_1dV_2
\end{align}
and the second order coefficient
\begin{align}
c_2 & = \frac{1}{2} \Biggl( 
 \sum_{i,j\in N_{gal}} \left( \left(\xi_{\hat{\delta} v}\frac{1}{\bm{\sigma}_v^2}  \xi_{v \hat{\delta}}\right)_{ij}
 + 2  \frac{\hat{\delta} v_j}{\sigma_{v,j}^2} \left(\xi_{\hat{\delta} v} \frac{1} {\bm{\sigma}_v^2} \xi_{vv}  \right)_{ij} + \frac{\hat{\delta} v_i}{\sigma_{v,i}^2} \left( \xi_{vv} \frac{1}{\bm{\sigma}_v^2} \xi_{vv}\right)_{ij} \frac{\hat{\delta} v_j}{\sigma_{v,j}^2}\right)\\
&- \sum_{i \in N_{gal}}\left( \left(  \xi_{vv} \frac{1}{\sigma_{v}^2}  \frac{1}{\sigma_{v}^2} \xi_{vv} \right)_{ii}  +2n \int \left( \left(\xi_{\hat{\delta} v}\frac{1}{\bm{\sigma}_v^2}  \xi_{v \hat{\delta}}\right)_{iV} + \frac{\hat{\delta} v_i}{\sigma_{v,i}^2} \left( \xi_{v v} \frac{1} {\bm{\sigma}_v^2} \xi_{v \hat{\delta}}  \right)_{iV} \right) dV  \right)\\
&-n  \int \left( \xi_{\hat{\delta} v} \frac{1}{\sigma_{v}}  \frac{1}{\sigma_{v}}  \xi_{v \hat{\delta}} \right)_{VV} dV  +n^2  \int  \left(\xi_{\hat{\delta} v}\frac{1}{\bm{\sigma}_v^2}  \xi_{v \hat{\delta}}\right)_{V_1V_2}dV_1dV_2 \Biggr).
\end{align}

In this second-order approximation of the log-likelihood, the minimum occurs at $A_{\text{min}}=-c_1/2c_2$ and the second derivative is $2c_2$.


END NEW STUFF

Given the enormous size of $\textbf{C}$, computing the log likelihood directly is not possible. In the following, we seek an approximation method accurate to first order in $A$. \par
First, we focus on the second term, $\hat{\delta} \textbf{y}^T\textbf{C}^{-1}\hat{\delta} \textbf{y}$. Looking at the definition of $\textbf{C}$ in equation (1), we want to factor $\textbf{C}_0$ into an identity matrix. To do so, we introduce matrix $\textbf{M}$ and $\hat{\delta} \textbf{y}$. We let $\textbf{M}=\sqrt{\textbf{C}_0}$, and $\textbf{S} = \textbf{M} \hat{\delta} \textbf{y} \textbf{M}$. Then $\hat{\delta} \textbf{y}=\textbf{M}^{-1}\textbf{S}\textbf{M}^{-1}$. Now, our matrix $\textbf{C}$ can be re-written in terms of $\hat{\delta} \textbf{y}$ and $\textbf{M}$:
\begin{equation}
\textbf{C} = \textbf{M}(\textbf{I}+A\hat{\delta} \textbf{y})\textbf{M},
\end{equation}
and its inverse, to first order, is
\[
\textbf{C}^{-1} = \textbf{M}^{-1}(1-A\hat{\delta} \textbf{y}) \textbf{M}^{-1} = -A \textbf{C}_0^{-1}\textbf{S}\textbf{C}_0^{-1} + \text{const}.
\] \par
Having this expression, we can now compute the second term in equation (2). Denoting the value of this term by $s$, we have
\begin{align*}
s &=\hat{\delta} \textbf{y}^T \textbf{C}^{-1}\hat{\delta} \textbf{y} \\
&= -A\left(\hat{\delta} \textbf{y}^T \textbf{C}_0^{-1}\right)\textbf{S}\left(\textbf{C}_0^{-1}\hat{\delta} \textbf{y}\right).
\end{align*} \par
Using the equations for covariance calculation from \cite{adam}, we can break the above expression into a sum of 6 terms and write out the explicit solution for each (apart from a zeroth-order constant). Let 
\[
-\left(\hat{\delta} \textbf{y}^T \textbf{C}_0^{-1}\right)\textbf{S}\left(\textbf{C}_0^{-1}\hat{\delta} \textbf{y}\right) = s_1 +s_2+\hdots +s_6
\]
\begin{align*}
s_1&= -\sum_{i,j\in N_{gal}} \xi_{\hat{\delta}\hat{\delta},ij},  \\
s_2 &= 2 \sum_{i\in N_{gal},j\in N_{cell}} \bar{N} \xi_{\hat{\delta}\hat{\delta},ij}, \\
&= 2\bar{N} \frac{b^2}{2\pi} \sum_{i\in N_{gal}} \int r^2drd\Omega  \int dk P(k) k^2 j_0 (k |\textbf{r}-\textbf{r}_i|), \\
s_3 &= -\sum_{i,j\in N_{cell}} \bar{N}^2  \xi_{\hat{\delta}\hat{\delta},ij}, \\ 
&=-\bar{N}^2 \frac{b^2}{2\pi} \int r_1^2dr_1d\Omega_1 \int r_2^2dr_2 d\Omega_2 \int dk P(k) k^2 j_0 (k |\textbf{r}_1-\textbf{r}_2|), \\
s_4 &=- \sum_{i,j\in N_{gal}} \frac{\hat{\delta} v_i}{\sigma_{v,i}^2} \xi_{vv,ij} \frac{\hat{\delta} v_j}{\sigma_{v,j}^2}, \\
s_5 &= -2 \sum_{i,j\in N_{gal}} \frac{\hat{\delta} v_j}{\sigma_{v,j}^2} \xi_{\hat{\delta} v,ij}, \\
s_6 &= 2\sum_{i\in N_{cell}, j\in N_{gal}} \bar{N}\frac{\hat{\delta} v_j}{\sigma_{v,j}^2} \xi_{\hat{\delta} v,ij},  \\
&= 2\bar{N}\sum_{j\in N_{gal}} \frac{aHfb\hat{\delta} v_j}{2\pi^2 \sigma^2_{v,j}}\int r^2 dr d\Omega \int dk P(k) k ((\hat{\textbf{r}}-\hat{\textbf{r}_j})\cdot\hat{\textbf{r}_j}) j_1(k|\textbf{r}-\textbf{r}_j|). 
%s_7 &= A  \sum_{j\in N_{cell}, i\in N_{gal}} \bar{N}\frac{\hat{\delta} v_i}{\sigma_{v,i}^2} \xi_{v\hat{\delta},ij}  \\
%&= A\bar{N} \sum_{i\in N_{gal}} \frac{+aHfb\hat{\delta} v_i}{2\pi^2 \sigma^2_{v,i}}\int r^2 dr d\Omega \int dk P(k) k (\textbf{r}\cdot\textbf{r}_i) j_1(k|\textbf{r}-\textbf{r}_i|)
\end{align*} \par
We note that in the above expressions, $P(k)$ is the power spectrum, $a$ the expansion factor, $H$ the Hubble constant, $b$ the galaxy bias, $f$ the growth rate of structure, and $j_l$ the spherical Bessel function. \par
Having computed the second term in equation (2), we now turn to the first term, $\ln\det\textbf{C}$. Using equation (3) to factor $\textbf{C}$, we have
\begin{align*}
\ln\det(\textbf{C}) &=\ln\det(\textbf{M}(\textbf{I}+A\textbf{X})\textbf{M})) \\
&= \ln\det(\textbf{I}+A\textbf{X}) +\text{const} \\
&\approx A \tr \textbf{X} \\
&= A\tr(\textbf{M}^{-1}\textbf{S}\textbf{M}^{-1}) \\
&= A\tr 
\left(
\begin{array}{cc}
\bar{N}\xi_{\hat{\delta}\hat{\delta}} & \frac{\sqrt{\bar{N}}}{\sigma_{v}} \xi_{\hat{\delta} v} \\
\frac{\sqrt{\bar{N}}}{\sigma_{v}} \xi_{v\hat{\delta}} & \frac{1}{\sigma_{v}^2} \xi_{vv} 
\end{array}
\right),
\end{align*}
The trace can be broken up into 2 sums, $d_1+d_2$:
\begin{align*}
d_1 = \frac{b^2\bar{N}}{2\pi} \int r^2drd\Omega  \int dk P(k) k^2 
\end{align*}
since $j_0(0)=1$.
\[
d_2 = \sum_{i}\xi_{vv,ii}\sigma_i^{-2}
\] \par
In the limit as $N_{cell}$ goes to infinity, $\bar{N}\propto 1/N_{cell}$ goes to zero. Therefore, terms that contain $k$ factors of $N_{cell}$ will vanish in the $N_{cell}\to\infty$ limit unless they also contain $k$ summations over $N_{cell}$: $\sum_{N_{cell}}$. The reason is that $\sum_{N_{cell}}\bar{N}$ stays finite when $N_{cell}\to \infty$. Looking at the above $s$ and $d$ terms, we see that all of them are finite in the large $\bar{N}$ limit, and the the full likelihood to first order in $\bar{N}$ is
\[
\boxed{A(s_1+s_2+s_4+s_5+s_6+d_1+d_2)}
\]\par
$s_3$ is not selected because it is quadratic in $\bar{N}$.
\section{Inversion to second order}
The $O(A^2)$ prefactor in $\textbf{C}^{-1}$ is 
\begin{align*}
\textbf{M}^{-1}\textbf{X}^2\textbf{M}^{-1} &= \textbf{M}^{-1} \textbf{M}^{-1} \textbf{S} \textbf{M}^{-1}\textbf{M}^{-1} \textbf{S} \textbf{M}^{-1}\textbf{M}^{-1} \\
&=\textbf{C}_0^{-1}\textbf{S}\textbf{C}_0^{-1}\textbf{S}\textbf{C}_0^{-1}
\end{align*}\par
I first sight I wanted to move $\textbf{S}$ across the $\textbf{C}_0^{-1}$ by using a commutator, thinking that it might simplify things a lot. But it turned out that the commutator did not lead to the expected cancelation, so let's do it the hard way. We first do $\textbf{C}_0^{-1}\textbf{S}\textbf{C}_0^{-1}\textbf{S}$, henceforth denoting the inverse of $\bm{\sigma}^2_v\textbf{I}$ as $\bm{\sigma}^{-1}$:
\begin{align*}
\textbf{C}_0^{-1}\textbf{S}\textbf{C}_0^{-1}\textbf{S} &=
\left(
\begin{array}{cc}
\bar{N}\xi_{\hat{\delta}\hat{\delta}} & \xi_{\hat{\delta} v} \\
\bar{N}\bm{\sigma}^{-1} \xi_{v\hat{\delta}} & \bm{\sigma}^{-1} \xi_{vv}
\end{array}
\right)
\left(
\begin{array}{cc}
\bar{N}\xi_{\hat{\delta}\hat{\delta}} & \xi_{\hat{\delta} v} \\
\bar{N}\bm{\sigma}^{-1} \xi_{v\hat{\delta}} & \bm{\sigma}^{-1} \xi_{vv}
\end{array}
\right) \\
&=
\left(
\begin{array}{cc}
\bar{N}^2 \xi_{\hat{\delta}\hat{\delta}}^2 + \bar{N}\xi_{\hat{\delta} v}\bm{\sigma}^{-1} \xi_{v\hat{\delta}} & \bar{N}\xi_{\hat{\delta}\hat{\delta}}\xi_{\hat{\delta} v}+\xi_{\hat{\delta} v}\bm{\sigma}^{-1}\xi_{vv} \\
\bar{N}^2\bm{\sigma}^{-1} \xi_{v\hat{\delta}}\xi_{\hat{\delta}\hat{\delta}}+\bar{N}\bm{\sigma}^{-1} \xi_{vv}\bm{\sigma}^{-1} \xi_{v\hat{\delta}} & \bar{N}\bm{\sigma}^{-1} \xi_{v\hat{\delta}}\xi_{\hat{\delta} v} + \bm{\sigma}^{-1} \xi_{vv}\bm{\sigma}^{-1} \xi_{vv}
\end{array}
\right)
\end{align*}\par
Next, we multiply $\textbf{C}_0^{-1}$ from the right and get
\begin{align*}
\textbf{C}_0^{-1}\textbf{S}\textbf{C}_0^{-1}\textbf{S}\textbf{C}_0^{-1} &=
\left(
\begin{array}{cc}
\bar{N}^2 \xi_{\hat{\delta}\hat{\delta}}^2 + \bar{N}\xi_{\hat{\delta} v}\bm{\sigma}^{-1} \xi_{v\hat{\delta}} & \bar{N}\xi_{\hat{\delta}\hat{\delta}}\xi_{\hat{\delta} v}+\xi_{\hat{\delta} v}\bm{\sigma}^{-1}\xi_{vv} \\
\bar{N}^2\bm{\sigma}^{-1} \xi_{v\hat{\delta}}\xi_{\hat{\delta}\hat{\delta}}+\bar{N}\bm{\sigma}^{-1} \xi_{vv}\bm{\sigma}^{-1} \xi_{v\hat{\delta}} & \bar{N}\bm{\sigma}^{-1} \xi_{v\hat{\delta}}\xi_{\hat{\delta} v} + \bm{\sigma}^{-1} \xi_{vv}\bm{\sigma}^{-1} \xi_{vv}
\end{array}
\right)
\left(
\begin{array}{cc}
\bar{N}^{-1}\textbf{I} & 0 \\
0 & \bm{\sigma}^{-1}
\end{array}
\right) \\
&=
\left(
\begin{array}{cc}
\bar{N} \xi_{\hat{\delta}\hat{\delta}}^2 + \xi_{\hat{\delta} v}\bm{\sigma}^{-1} \xi_{v\hat{\delta}} & \bar{N}\xi_{\hat{\delta}\hat{\delta}}\xi_{\hat{\delta} v}\bm{\sigma}^{-1}+\xi_{\hat{\delta} v}\bm{\sigma}^{-1}\xi_{vv}\bm{\sigma}^{-1} \\
\bar{N}\bm{\sigma}^{-1} \xi_{v\hat{\delta}}\xi_{\hat{\delta}\hat{\delta}}+\bm{\sigma}^{-1} \xi_{vv}\bm{\sigma}^{-1} \xi_{v\hat{\delta}} & \bar{N}\bm{\sigma}^{-1} \xi_{v\hat{\delta}}\xi_{\hat{\delta} v}\bm{\sigma}^{-1} + \bm{\sigma}^{-1} \xi_{vv}\bm{\sigma}^{-1} \xi_{vv}\bm{\sigma}^{-1}
\end{array}
\right)
\end{align*}\par
Let's denote the 4 blocks in the above matrix by
\[
\textbf{C}_0^{-1}\textbf{S}\textbf{C}_0^{-1}\textbf{S}\textbf{C}_0^{-1}=
\left(
\begin{array}{cc}
\textbf{A} & \textbf{B} \\
\textbf{C} & \textbf{D}
\end{array}
\right)
\]\par
Let's now compute each block (the summations below are taken from $0$ to $N_{gal}$ unless otherwise specified). 
\begin{align*}
\textbf{A}_{ij} &=(\bar{N} \xi_{\hat{\delta}\hat{\delta}}^2 + \xi_{\hat{\delta} v}\bm{\sigma}^{-1} \xi_{v\hat{\delta}})_{ij} \\
&= \frac{\bar{N}b^4}{4\pi^4} \int r^2drd\Omega \iint dk_1dk_2 P(k_1)P(k_2) k_1^2k_2^2j_0(k_1|\textbf{r}-\textbf{r}_i|)j_0(k_2|\textbf{r}-\textbf{r}_j|) +\sum_{m}\sigma_{m}^{-2}\xi_{\hat{\delta} v,im}\xi_{v\hat{\delta},mj} \\
&=\frac{\bar{N}b^4}{4\pi^4} \int r^2drd\Omega \iint dk_1dk_2 P(k_1)P(k_2) k_1^2k_2^2j_0(k_1|\textbf{r}-\textbf{r}_i|)j_0(k_2|\textbf{r}-\textbf{r}_j|)\\
&+\frac{a^2H^2f^2b^2}{4\pi^2} \sum_m \sigma_{m}^{-2} \iint dk_1dk_2 P(k_1)P(k_2) k_1k_2j_1(k_1|\textbf{r}_i-\textbf{r}_m|)j_1(k_2|\textbf{r}_m-\textbf{r}_j|)(\hat{\textbf{r}}_i-\hat{\textbf{r}}_m)\cdot\hat{\textbf{r}}_m(\hat{\textbf{r}}_m-\hat{\textbf{r}}_j)\cdot\hat{\textbf{r}}_m\\
\textbf{B}_{ij} &= \frac{\bar{N}aHfb^3\sigma_j^{-2}}{4\pi^4}\int r^2drd\Omega \iint dk_1dk_2 P(k_1)P(k_2)k_1^2k_2(\hat{\textbf{r}}-\hat{\textbf{r}}_j)\cdot\hat{\textbf{r}}_jj_0(k_1|\textbf{r}-\textbf{r}_i|)j_1(k_2|\textbf{r}-\textbf{r}_j|)\\
&+ \sigma_j^{-2}\sum_m \sigma_m^{-2}\xi_{\hat{\delta} v,im}\xi_{vv,mj} \\
&=\frac{\bar{N}aHfb^3\sigma_j^{-2}}{4\pi^4}\int r^2drd\Omega \iint dk_1dk_2 P(k_1)P(k_2)k_1^2k_2(\hat{\textbf{r}}-\hat{\textbf{r}}_j)\cdot\hat{\textbf{r}}_jj_0(k_1|\textbf{r}-\textbf{r}_i|)j_1(k_2|\textbf{r}-\textbf{r}_j|) \\
&+\frac{\sigma_j^{-2}aHfb}{2\pi^2}\sum_m \sigma_m^{-2} \xi_{vv,mj}\int dk P(k)k j_1(k|\textbf{r}_i-\textbf{r}_m|)(\hat{\textbf{r}}_i-\hat{\textbf{r}}_m)\cdot\hat{\textbf{r}}_m \\
\textbf{C}_{ij} &=\frac{\bar{N}aHfb^3\sigma_i^{-2}}{4\pi^4}\int r^2drd\Omega \iint dk_1dk_2 P(k_1)P(k_2)k_1k_2^2(\hat{\textbf{r}_i}-\hat{\textbf{r}})\cdot\hat{\textbf{r}}_ij_0(k_2|\textbf{r}-\textbf{r}_j|)j_1(k_1|\textbf{r}-\textbf{r}_i|)\\
&+ \sigma_j^{-2}\sum_m \sigma_m^{-2}\xi_{vv,im}\xi_{v\hat{\delta},mj} \\
&=\frac{\bar{N}aHfb^3\sigma_i^{-2}}{4\pi^4}\int r^2drd\Omega \iint dk_1dk_2 P(k_1)P(k_2)k_1k_2^2(\hat{\textbf{r}_i}-\hat{\textbf{r}})\cdot\hat{\textbf{r}}_ij_0(k_2|\textbf{r}-\textbf{r}_j|)j_1(k_1|\textbf{r}-\textbf{r}_i|)\\
&+\frac{\sigma_j^{-2}aHfb}{2\pi^2}\sum_m \sigma_m^{-2} \xi_{vv,im}\int dk P(k)k j_1(k|\textbf{r}_m-\textbf{r}_j|)(\hat{\textbf{r}}_m-\hat{\textbf{r}}_j)\cdot\hat{\textbf{r}}_m\\
\textbf{D}_{ij} &=\frac{\bar{N}a^2H^2f^2b^2\sigma_i^{-2}\sigma_j^{-2}}{4\pi^4} \int r^2drd\Omega \iint dk_1dk_2 P(k_1)P(k_2)k_1^2k_2^2(\hat{\textbf{r}_i}-\hat{\textbf{r}})\cdot\hat{\textbf{r}}_i(\hat{\textbf{r}}-\hat{\textbf{r}}_j)\cdot\hat{\textbf{r}}_j\times \\ 
&\times j_1(k_1|\textbf{r}-\textbf{r}_i|)j_1(k_2|\textbf{r}-\textbf{r}_j|) +\sigma_i^{-2}\sigma_j^{-2}\sum_m \sigma_m^{-2}\xi_{vv,im}\xi_{vv,mj}
\end{align*}\par
Having obtained the 4 block matrices, we now proceed with the calculation of $\textbf{x}^T\textbf{C}_0^{-1}\textbf{S}\textbf{C}_0^{-1}\textbf{S}\textbf{C}_0^{-1}\textbf{x}$. Similar to what we did for the first order, we break the sum into the following terms (again, summation is within the range of $N_{gal}$ by default unless otherwise stated):
\begin{align*}
S_1 &= \sum_{i, j} \textbf{A}_{ij} \\
&=\frac{\bar{N}b^4}{4\pi^4}\sum_{i, j} \int r^2drd\Omega \iint dk_1dk_2 P(k_1)P(k_2) k_1^2k_2^2j_0(k_1|\textbf{r}-\textbf{r}_i|)j_0(k_2|\textbf{r}-\textbf{r}_j|)  \\
&+ \frac{a^2H^2f^2b^2}{4\pi^2} \sum_{m,i,j} \sigma_{m}^{-2} \iint dk_1dk_2 P(k_1)P(k_2) k_1k_2j_1(k_1|\textbf{r}_i-\textbf{r}_m|)j_1(k_2|\textbf{r}_m-\textbf{r}_j|)(\hat{\textbf{r}}_i-\hat{\textbf{r}}_m)\cdot\hat{\textbf{r}}_m(\hat{\textbf{r}}_m-\hat{\textbf{r}}_j)\cdot\hat{\textbf{r}}_m\\
S_2 &=-2\bar{N} \sum_{i\in N_{gal},j\in N_{cell}} \textbf{A}_{ij}\\
&=\frac{-\bar{N}^2 b^4}{2\pi^4}\sum_{i}\int r^2r_j^2drd\Omega dr_j d\Omega_j \iint dk_1dk_2 P(k_1)P(k_2) k_1^2k_2^2j_0(k_1|\textbf{r}-\textbf{r}_i|)j_0(k_2|\textbf{r}-\textbf{r}_j|)-\frac{a^2H^2f^2b^2\bar{N}}{2\pi^2}\times \\
&\times  \sum_{m,i} \sigma_{m}^{-2}\int r_j^2dr_j d\Omega_j \iint dk_1dk_2 P(k_1)P(k_2) k_1k_2j_1(k_1|\textbf{r}_i-\textbf{r}_m|)j_1(k_2|\textbf{r}_m-\textbf{r}_j|)(\hat{\textbf{r}}_i-\hat{\textbf{r}}_m)\cdot\hat{\textbf{r}}_m \times \\
&\times (\hat{\textbf{r}}_m-\hat{\textbf{r}}_j)\cdot\hat{\textbf{r}}_m \\
S_3 &= \bar{N}^2 \sum_{i, j\in N_{cell}} \textbf{A}_{ij} \\
&=\frac{\bar{N}^3b^4}{4\pi^4}\int r^2r_i^2r_j^2 drd\Omega d\Omega_i d\Omega_j \iint dk_1dk_2 P(k_1)P(k_2) k_1^2k_2^2j_0(k_1|\textbf{r}-\textbf{r}_i|)j_0(k_2|\textbf{r}-\textbf{r}_j|)  \\
&+ \frac{a^2H^2f^2b^2 \bar{N}^2}{4\pi^2} \sum_{m} \sigma_{m}^{-2}\int r_i^2r_j^2 dr_idr_j d\Omega_i d\Omega_j  \iint dk_1dk_2 P(k_1)P(k_2) k_1k_2j_1(k_1|\textbf{r}_i-\textbf{r}_m|)j_1(k_2|\textbf{r}_m-\textbf{r}_j|)\times\\
&\times (\hat{\textbf{r}}_i-\hat{\textbf{r}}_m)\cdot\hat{\textbf{r}}_m(\hat{\textbf{r}}_m-\hat{\textbf{r}}_j)\cdot\hat{\textbf{r}}_m \\
S_4 &= \sum_{i,j} \textbf{D}_{ij}\hat{\delta} v_i \hat{\delta} v_j \\
&= \frac{\bar{N}a^2H^2f^2b^2}{4\pi^4}\sum_{i,j}\sigma_i^{-2}\sigma_j^{-2}\hat{\delta} v_i \hat{\delta} v_j \int r^2drd\Omega \iint dk_1dk_2 P(k_1)P(k_2)k_1^2k_2^2(\hat{\textbf{r}_i}-\hat{\textbf{r}})\cdot\hat{\textbf{r}}_i(\hat{\textbf{r}}-\hat{\textbf{r}}_j)\cdot\hat{\textbf{r}}_j\times \\ 
&\times j_1(k_1|\textbf{r}-\textbf{r}_i|)j_1(k_2|\textbf{r}-\textbf{r}_j|) +\sum_{m,i,j} \sigma_i^{-2}\sigma_j^{-2}\sigma_m^{-2}\hat{\delta} v_i \hat{\delta} v_j\xi_{vv,im}\xi_{vv,mj}\\
S_5 &= 2\sum_{i,j}\textbf{B}_{ij}\hat{\delta} v_j \\
&=\frac{\bar{N}aHfb^3}{2\pi^4}\sum_{i,j}\sigma_j^{-2}\hat{\delta} v_j\int r^2drd\Omega \iint dk_1dk_2 P(k_1)P(k_2)k_1^2k_2(\hat{\textbf{r}}-\hat{\textbf{r}}_j)\cdot\hat{\textbf{r}}_jj_0(k_1|\textbf{r}-\textbf{r}_i|)j_1(k_2|\textbf{r}-\textbf{r}_j|) \\
&+\frac{\sigma_j^{-2}aHfb}{\pi^2}\sum_{m,i,j}\hat{\delta} v_j \sigma_m^{-2} \xi_{vv,mj}\int dk P(k)k j_1(k|\textbf{r}_i-\textbf{r}_m|)(\hat{\textbf{r}}_i-\hat{\textbf{r}}_m)\cdot\hat{\textbf{r}}_m\\
S_6 &=-2\bar{N} \sum_{i\in N_{cell},j\in N_{gal}}\textbf{B}_{ij} \hat{\delta} v_j\\
&=\frac{-\bar{N}^2aHfb^3}{2\pi^4}\sum_{j}\sigma_j^{-2}\hat{\delta} v_j\int r^2r_i^2drdr_id\Omega d\Omega_i \iint dk_1dk_2 P(k_1)P(k_2)k_1^2k_2(\hat{\textbf{r}}-\hat{\textbf{r}}_j)\cdot\hat{\textbf{r}}_jj_0(k_1|\textbf{r}-\textbf{r}_i|)\times \\
&\times j_1(k_2|\textbf{r}-\textbf{r}_j|)-\frac{\bar{N}aHfb}{\pi^2}\sum_{m,j}\sigma_j^{-2} \hat{\delta} v_j \sigma_m^{-2} \xi_{vv,mj}\int r_i^2 dr_id\Omega_i\int dk P(k)k j_1(k|\textbf{r}_i-\textbf{r}_m|)(\hat{\textbf{r}}_i-\hat{\textbf{r}}_m)\cdot\hat{\textbf{r}}_m
\end{align*}\par
The seconder-order contribution from $\textbf{x}^T\textbf{C}^{-1}\textbf{x}$ to the likelihood is thus
\[
A^2 (S_1 +S_2 +S_3 +S_4 +S_5 +S_6)
\]\par
\section{Log det to second order}
We now calculate the second-order contribution from the log det term. The $O(A^2)$ prefactor in $\ln \det(\textbf{I}+A\textbf{X})$ is
\[
-\frac{1}{2}\tr (\textbf{X}^2) = -\frac{1}{2}\tr(\textbf{M}^{-1}\textbf{S}\textbf{M}^{-1}\textbf{M}^{-1}\textbf{S}\textbf{M}^{-1})
\]\par
Since the trace is cyclic, we can permute the matrix multiplication and make it into something we've calculated on page 3:
\begin{align*}
-\frac{1}{2}\tr (\textbf{X}^2)&=-\frac{1}{2}\tr(\textbf{M}^{-1}\textbf{S}\textbf{M}^{-1}\textbf{M}^{-1}\textbf{S}\textbf{M}^{-1})\\
&=-\frac{1}{2}\tr(\textbf{M}^{-1}\textbf{M}^{-1}\textbf{S}\textbf{M}^{-1}\textbf{M}^{-1}\textbf{S}) \\
&= -\frac{1}{2}\tr(\textbf{C}_0^{-1}\textbf{S}\textbf{C}_0^{-1}\textbf{S}) \\
&= -\frac{1}{2}\tr\left\{
\left(
\begin{array}{cc}
\bar{N}^2 \xi_{\hat{\delta}\hat{\delta}}^2 + \bar{N}\xi_{\hat{\delta} v}\bm{\sigma}^{-1} \xi_{v\hat{\delta}} & \bar{N}\xi_{\hat{\delta}\hat{\delta}}\xi_{\hat{\delta} v}+\xi_{\hat{\delta} v}\bm{\sigma}^{-1}\xi_{vv} \\
\bar{N}^2\bm{\sigma}^{-1} \xi_{v\hat{\delta}}\xi_{\hat{\delta}\hat{\delta}}+\bar{N}\bm{\sigma}^{-1} \xi_{vv}\bm{\sigma}^{-1} \xi_{v\hat{\delta}} & \bar{N}\bm{\sigma}^{-1} \xi_{v\hat{\delta}}\xi_{\hat{\delta} v} + \bm{\sigma}^{-1} \xi_{vv}\bm{\sigma}^{-1} \xi_{vv}
\end{array}
\right)
\right\}
\end{align*}\par
Expanding the trace, we get (using $D_i$ to label each term)
\begin{align*}
D_1 + D_2 + D_3 + D_4 &\equiv -\frac{\bar{N}^2}{2} \sum_{i,j\in N_{cell}} \xi_{\hat{\delta}\hat{\delta},ij}\xi_{\hat{\delta}\hat{\delta},ji} -\frac{\bar{N}}{2}\sum_{i \in N_{cell},j\in N_{gal}} \sigma_j^{-2}\xi_{\hat{\delta} v,ij} \xi_{v\hat{\delta},ji} \\
&-\frac{\bar{N}}{2} \sum_{i\in N_{gal},j\in N_{cell}} \sigma_i^{-2} \xi_{v\hat{\delta},ij}\xi_{\hat{\delta} v,ji} - \frac{1}{2} \sum_{i,j} \sigma_i^{-2}\sigma_j^{-2} \xi_{vv,ij}\xi_{vv,ji}
\end{align*}\par
Let's calculate each $D$:
\begin{align*}
D_1 &= -\frac{\bar{N}^2 b^4}{8\pi^4} \int r^2r_i^2drdr_id\Omega d\Omega_i d \iint dk_1dk_2 P(k_1)P(k_2) k_1^2k_2^2j_0(k_1|\textbf{r}-\textbf{r}_i|)j_0(k_2|\textbf{r}-\textbf{r}_i|) \\
D_2 &= -\frac{a^2H^2f^2b^2\bar{N}}{8\pi^2} \sum_j \sigma_{j}^{-2}\int r_i^2 dr_i d\Omega_i \iint dk_1dk_2 P(k_1)P(k_2) k_1k_2j_1(k_1|\textbf{r}_i-\textbf{r}_j|)j_1(k_2|\textbf{r}_j-\textbf{r}_i|)\times \\
&\times (\hat{\textbf{r}}_i-\hat{\textbf{r}}_j)\cdot\hat{\textbf{r}}_j(\hat{\textbf{r}}_j-\hat{\textbf{r}}_i)\cdot\hat{\textbf{r}}_j\\
D_3 &= -\frac{\bar{N}a^2H^2f^2b^2}{8\pi^4} \sum_{i}\sigma_i^{-2} \int r^2drd\Omega \iint dk_1dk_2 P(k_1)P(k_2)k_1^2k_2^2(\hat{\textbf{r}_i}-\hat{\textbf{r}})\cdot\hat{\textbf{r}}_i(\hat{\textbf{r}}-\hat{\textbf{r}}_i)\cdot\hat{\textbf{r}}_i\times \\ 
&\times j_1(k_1|\textbf{r}-\textbf{r}_i|)j_1(k_2|\textbf{r}-\textbf{r}_i|)\\
D_4 &=- \frac{1}{2} \sum_{i,j} \sigma_i^{-2}\sigma_j^{-2} \xi_{vv,ij}\xi_{vv,ji}
\end{align*}\par
By similar argument as we did at the end of the previous section, all of the above stay finite in the limit $N_{cell}\to\infty$. In summary, the full likelihood at second order (in $A$, not in $\bar{N}$) is
\[
\boxed{A^2 (S_1 +S_2' +S_4 +S_5 +S_6'+D_2+D_3+D_4)}
\]
where
\begin{align*}
S_2' &=-\frac{a^2H^2f^2b^2\bar{N}}{2\pi^2}\times \\
&\times  \sum_{m,i} \sigma_{m}^{-2}\int r_j^2dr_j d\Omega_j \iint dk_1dk_2 P(k_1)P(k_2) k_1k_2j_1(k_1|\textbf{r}_i-\textbf{r}_m|)j_1(k_2|\textbf{r}_m-\textbf{r}_j|)(\hat{\textbf{r}}_i-\hat{\textbf{r}}_m)\cdot\hat{\textbf{r}}_m \times \\
&\times (\hat{\textbf{r}}_m-\hat{\textbf{r}}_j)\cdot\hat{\textbf{r}}_m \\
S_6' &=-\frac{\bar{N}aHfb}{\pi^2}\sum_{m,j}\sigma_j^{-2} \hat{\delta} v_j \sigma_m^{-2} \xi_{vv,mj}\int r_i^2 dr_id\Omega_i\int dk P(k)k j_1(k|\textbf{r}_i-\textbf{r}_m|)(\hat{\textbf{r}}_i-\hat{\textbf{r}}_m)\cdot\hat{\textbf{r}}_m
\end{align*}
after removing the $O(\bar{N}^2)$ terms. Note that $S_3$ and $D_1$ dropped out completely.
\section{Full likelihood to second order}
Adding up both the first and the second order contribution, we have
\[
\mathcal{L} =\boxed{ A^2 (S_1 +S_2'  +S_4 +S_5 +S_6'+D_2+D_3+D_4) +A(s_1+s_2+s_4+s_5+s_6+d_1+d_2)}
\]
with extremum at
\[
\boxed{A^*= -\frac{s_1+s_2+s_4+s_5+s_6+d_1+d_2}{2(S_1 +S_2'  +S_4 +S_5 +S_6'+D_2+D_3+D_4)}}
\]
\end{comment}

\appendix
\section{Estimator and Fisher Matrix}
\label{appendix:sec}
Suppose we have a likelihood that has parameters $\theta$ and $p$ such that
$$f\equiv-\ln{\mathcal{L}(\theta,p)} = \frac{1}{2}\left(\ln{(\det(C(\theta)))} + (y-y_0 )^T C^{-1} (y-y_0)\right),$$
where $C$ is invertible (the covariance matrix is positive definite),
$$y_0 = p_\alpha x_\alpha,$$ and
 $$C=C_0 + \theta_\alpha S_\alpha.$$
Then
\begin{align}
\frac{df}{d\theta_\alpha} &= \frac{1}{2}\left(\tr(C^{-1}S_\alpha)   - (y-y_0)^T  C^{-1} S_{\alpha} C^{-1}  (y-y_0)\right)\\ 
\frac{df}{dp_{\alpha}} &= -(y-y_0)^T C^{-1} x_\alpha\\ 
\frac{df^2}{d\theta_{\alpha} d\theta_\beta} &=- \frac{1}{2}  \tr (C^{-1} S_\alpha C^{-1}  S_\beta) +  (y-y_0)^T (
C^{-1} S_{\alpha} C^{-1} S_\beta C^{-1}
)
(y-y_0) \\ 
\frac{df^2}{dp_{\alpha} dp_\beta} &= x^T_\alpha C^{-1} x_\beta\\ 
\frac{df^2}{d\theta_\alpha\,dp_\beta} &= (y-y_0)^T C^{-1}  S _\alpha C^{-1} x_\beta.
\end{align}
For a Gaussian distribution
\begin{align}
F_{\alpha \beta}^{(\theta \theta)} & = \left \langle  \frac{df^2}{d\theta_{\alpha} d\theta_\beta} \right \rangle =  -\frac{1}{2} \tr (C^{-1} S_\alpha C^{-1}  S_\beta)    + \tr (C^{-1} S_\alpha C^{-1}  S_\beta) \\ 
& = \frac{1}{2} \tr (C^{-1} S_\alpha C^{-1}  S_\beta) \\
F_{\alpha \beta}^{(p p)}  & = \left \langle  \frac{df^2}{dp_{\alpha} dp_\beta} \right \rangle=  x^T_\alpha C^{-1} x_\beta\\ 
F_{\alpha \beta}^{(\theta p)}  & = \left \langle  \frac{df^2}{d\theta_\alpha \,dp_\beta} \right \rangle =0,
\end{align}
where the ensemble averages in the Fisher matrices satisfy  $\langle y-y_0 \rangle = 0$, $\langle (y-y_0) (y-y_0)^T\rangle = C$.

We are interested in the properties of the function at its minimum, i.e. at $(\hat{\theta},\hat{p})$ occurs where $df/d\theta_\alpha=0 $ and $df/dp{_\alpha}=0$.
\begin{align}
0 & = \frac{1}{2}\left(\tr(C^{-1}S_\alpha C^{-1} C)  - (y-y_0)^T  C^{-1} S_{\alpha} C^{-1}  (y-y_0)\right) \\
& = \frac{1}{2}\left(\tr(C^{-1}S_\alpha C^{-1} (C_0 + \hat{\theta}_\beta S_\beta) )  - (y-y_0)^T  C^{-1} S_{\alpha} C^{-1}  (y-y_0)\right) \\
& =\hat{ \theta}_\beta F^{(\theta \theta)}_{\alpha \beta} +  \frac{1}{2}\left(\tr(C^{-1}S_\alpha C^{-1} C_0)  - (y-y_0)^T  C^{-1} S_{\alpha} C^{-1}  (y-y_0)\right) \\
\hat{ \theta}_\beta  & =  \frac{1}{2} F^{(\theta \theta)\, -1}_{\beta \alpha} \left( (y-y_0)^T  C^{-1} S_{\alpha} C^{-1}  (y-y_0) - \tr(C^{-1}S_\alpha C^{-1} C_0) \right) \\
0 &=  (y-\hat{p}_\beta x_\beta )^T C^{-1} x_\alpha \\
& = y^TC^{-1} x_\alpha - \hat{p}_\beta F^{(pp)}_{\alpha \beta} \\
 \hat{p}_\beta  & =   F^{(pp)\,-1}_{\beta \alpha} (y^TC^{-1} x_\alpha ).
\end{align}

\newpage
\section{More Complete Model}


The model consists of top-level parameters
\begin{itemize}
\item $\gamma$: Growth index;
\item $\Omega_{M0}$: Mass density;
\item $b$: Host-galaxy bias;
\item $\sigma^2_{v,\text{NL}}$: Contribution of velocities due to overdensities of scales $k>k_{v,\text{max}}$;
\item $\sigma^2_{\rho,\text{NL}}$: Overdensities of scales $k>k_{\rho,\text{max}}$;
\item $\mathcal{M} = M -5\log{H_0}$: The zeropoint of the SN Hubble diagram;
\item $\sigma_M$: SN~Ia intrinsic magnitude dispersion.
\end{itemize}
For shorthand these are collectively labeled with $\theta$.

The model latent parameters are
\begin{itemize}
\item $\bar{\delta} | \gamma, \Omega_{M0}$: Fourier modes that describe the linear density field through $P_{\delta \delta}(f\sigma_8(\gamma, \Omega_{M0}))$;
\item $\delta | \bar{\delta}, \sigma^2_{\rho,\text{NL}}$: The density field including non-linear density effects;
\item $\delta_g | \delta, b$: The galaxy density field;
\item $v | \delta$: The velocity field excluding non-linear velocity effects;
\item $\{u_\alpha, \mu_\alpha + 5\log{H_0} \}| \delta_g $: Comoving coordinates for the  SNe in the universe;
\item $\Delta M_\alpha | \sigma_M$: Absolute magnitude difference from the average for SN $\alpha$;
\item $\Delta v_\alpha| \sigma^2_{v,\text{NL}}$:  Non-linear peculiar velocity for SN $\alpha$;
\end{itemize}
From these parameters we have the expected values of the SN observables
\begin{align*}
m_\alpha &=  (\mu_\alpha + 5\log{H_0})+\mathcal{M} + \Delta M_\alpha\\
z_\alpha &= f(\mu_\alpha,v_\alpha+\Delta v_\alpha).
%n_\alpha(u,m)& = n_0 \sum_\alpha \delta(m_\alpha,z_\alpha).
\end{align*}
%There is another set of latent parameters for a set of ``virtual'' SNe.
%\begin{itemize}
%\item $\{u_\alpha, \mu_\alpha + 5\log{H_0}\} | \delta_g$: True comoving coordinates for the set of SN;;
%\item $\Delta M_\alpha | \sigma_M$: Absolute magnitude difference from the average for SN $\alpha$;
%\end{itemize}
For shorthand these are collectively called $\phi$.

The data is a set of supernovae.  The data for supernova $i$ are 
\begin{itemize}
\item $\hat{u}_i $: Angular coordinates RA and Dec;
\item $\hat{m}_i $: peak magnitude (proxy for distance);
\item $\hat{z}_i $: redshift;
\end{itemize}

I have a model where the probability of the data is
\begin{align}
p(\bar{\delta}|\gamma,\Omega_{M0})  p(\delta| \bar{\delta},\sigma^2_{\rho,\text{NL}})  p(\delta_g|\delta,b) p(v|\delta)\\
p(\{u_\alpha, \mu_\alpha + 5\log{H_0}\} | \delta_g) \\
p(\{\Delta M_\alpha\} | \sigma_M) p(\{m_\alpha\} | \{\mu_\alpha + 5\log{H_0}\},\{\Delta M_\alpha\},\mathcal{M})\\
    p(\{\Delta v_\alpha\} | \sigma^2_v)
     p(\{z_\alpha\}| \{\mu_\alpha\}, v, \{\Delta v_\alpha\})\\
 %    p(\{\hat{u}_\alpha\}| \{u_\alpha\}) p(\{\hat{m}_\alpha\}| \{m_\alpha\})  \\
     p( \{\hat{u}_i\}, \{\hat{m}_i\}  | \{u_\alpha\}, \{m_\alpha\})       p(\{\hat{z}_i\}| \{z_\alpha\}).
\end{align}
The first line describes the fields, the second the SNe, the third their apparent magnitudes, the fourth apparent redshifts, and the last
connects the model observables with their experimental realizations.
The $p( \{\hat{u}_i\}, \{\hat{m}_i\}  | \{u_\alpha\}, \{m_\alpha\})$ does not depend on ordering.

The term $p(\{u_\alpha, \mu_\alpha + 5\log{H_0}\} | \delta_g)$ is the probability of the appearance of objects at certain coordinates and non-appearance elsewhere.
It is convenient to rewrite this term as
\begin{align}
p(\{u_\alpha, \mu_\alpha + 5\log{H_0}\} | \delta_g) & = p(0| \delta_g) \prod_\alpha \frac{ p(u_\alpha, \mu_\alpha + 5\log{H_0} | \delta_g)}{p(0_\alpha| \delta_g)}  \\
& = e^{-{\bar{N}}}  \prod_\alpha \frac{\bar{n}_\alpha^{n_\alpha}}{n_\alpha!}
\end{align}
where $\bar{N}$ is the expected number of discovered SN, $\bar{n}$ is the expected number of SN discovered at position $\alpha$, and
$n_\alpha$ is the number of SNe found at the same $\alpha$ position.  We expect $\bar{n} \propto \delta_g$.

The survey has a selection function that is dependent on the angular position and observed magnitude $S(u,m)$.
It is thus straightforward to include the selection when calculating $p( \{\hat{u}_i\}, \{\hat{m}_i\}  | \{u_\alpha\}, \{m_\alpha\})$.


\newpage


The average magnitude distance modulus of the bin is $\mu_\beta$ and the bin has linear radial peculiar velocity $v_\beta$.
Each SN~Ia in that space is parameterized by $\Delta M_i$ and $\Delta v_i$.  We assume RA and Dec are measured
perfectly, e.g. no lensing.
\section{Model}

Suppose we have two sets of data, the number of galaxies $N$ in a volume
bin and the radial velocities of galaxies $v$.  These may or may not be the same galaxies.
We also assume that the universe is periodic with a minimum length scale, meaning that fields can be
described by a finite number of Fourier modes.  Given a model of gravity, we can deterministically calculate the
velocity field given the density field.
The likelihood is then
\begin{align}
L(\theta; N, v) & =p(N, v| \theta) \\
& = \int p(N,v,\delta | \theta) d\delta \\\
& = p(v|\theta)  \int p(N | \delta) p(\delta| v, \theta) d\delta \\
& = p(v|\theta)  \int \prod_{i=1}^{N_{bin}}  p(N_i | \delta) p(\delta| v, \theta) d\delta \\
& = p(v|\theta)  \int \prod_{i=1}^{N_{bin}}  \sum_{\tilde{N}_i =N_i}^\infty    p(N_i | \tilde{N}_i) p(\tilde{N}_i | \delta)  p(\delta| v, \theta) d\delta \\
& = p(v|\theta)  \int \prod_{i=1}^{N_{bin}}  \sum_{\tilde{N}_i =N_i}^\infty    \mathcal{B}(N_i; \tilde{N}_i, p_i) \mathcal{P}(\tilde{N}_i ; N_{0i}(\delta))  p(\delta| v, \theta) d\delta \\
& =  p(v|\theta)  \int  \exp \left[ \sum_{i=1}^{N_{bin}} \ln{  \left( \sum_{\tilde{N}_i =N_i}^\infty  \mathcal{B}(N_i; \tilde{N}_i, p_i) \mathcal{P}(\tilde{N}_i ; N_{0i}(\delta))\right)} \right]  p(\delta| v, \theta) d\delta 
%&= \int  p(N | \delta)  p(v | u)  p(\delta,v| \theta) d\delta\, du\\ 
%& = \int  \prod_{i=1}^{N_{bin}}   p(N_i | \delta_i)p(v| u) p(\delta, u | \theta)d\delta\, du\\ 
%& = \int  \prod_{i=1}^{N_{bin}}  \sum_{\tilde{N}_i =N_i}^\infty  p(N_i | \tilde{N}_i) p(\tilde{N}_i | \delta) p(v| u) p(\delta, u | \theta) d\delta\, du \\
%& = \int  \prod_{i=1}^{N_{bin}}  \sum_{\tilde{N}_i =N_i}^\infty  \mathcal{B}(N_i; \tilde{N}_i, p_i) \mathcal{P}(\tilde{N}_i ; N_{0i}(\delta))p(v| u) p(\delta, u | \theta) d\delta\, du \\
%&= \int  \exp \left[ \sum_{i=1}^{N_{bin}} \ln{  \left( \sum_{\tilde{N}_i =N_i}^\infty  \mathcal{B}(N_i; \tilde{N}_i, p_i) \mathcal{P}(\tilde{N}_i ; N_{0i}(\delta))\right)} \right]  p(v| u) p(\delta, u | \theta) d\delta\, du
\end{align}
Here $\delta$ and $i$ are  the mass-density  and momentum fields at the bin/galaxy positions, $\tilde{N}_i$ is the latent parameter
for the  total number of galaxies in bin $i$, $N_{0i}(\hat{\delta})$ is the expected
average number of galaxies in bin $i$, and $p_i$ is the probability of for a galaxy in cell $i$ to enter the sample.
$\mathcal{B}$ is the binomial distribution and $\mathcal{P}$ is the Poisson distribution.


The summation over bins an be split up into two terms assuming there are no cells with greater than two galaxies.
The first is a term as if there were zero galaxies in all bins, $N_i=0\ \forall i$.
\begin{align}
 \sum_{i=1}^{N_{bin}} \ln{  \left( \sum_{\tilde{N}_i =N_i}^\infty  \mathcal{B}(0; \tilde{N}_i, p_i) \mathcal{P}(\tilde{N}_i ; N_{0i}(\delta))\right)} & =
\sum_{i=1}^{N_{bin}} \ln{  \left( \sum_{\tilde{N}_i =0}^\infty   (1-p_i)^{ \tilde{N}_i} \frac{N_{0i}^{\tilde{N}_i} e^{-N_{0i}}}{\tilde{N}_i!} \right)}\\ 
& = \sum_{i=1}^{N_{bin}} (1-p_i)N_{0i} - N_{0i}\\
& = - \sum_{i=1}^{N_{bin}} p_i N_{0i}  \\
& = - \int  p(x) n_{0}(x) d^3x
\end{align}
where in the limit of  infinitesimally small bins $N_{0}$ approaches zero and
the integral replaces the summation.

Consider a cell with $N_i$ galaxies.  It contributes a 
\begin{align}
\ln{  \left( \sum_{\tilde{N}_i =N_i}^\infty  \mathcal{B}(N_i; \tilde{N}_i, p_i) \mathcal{P}(\tilde{N}_i ; N_{0i}(\delta))\right)}\\
= \ln{  \left( \sum_{\tilde{N}_i =N_i}^\infty \frac{\tilde{N}_i!}{N_i!(\tilde{N}_i-N_i)!}   p_i^{N_i} (1-p_i)^{ \tilde{N}_i-N_i} \frac{N_{0i}^{\tilde{N}_i} e^{-N_{0i}}}{\tilde{N}_i!} \right)}
 \\
 = \ln{  \left( \frac{(p_iN_{0i})^N_ie^{-p_iN_{0i}}}{N_i!}  \right)} \\
 =  N_i \ln{\left(  p_i  N_{0i} \right) } -p_i N_{0i} - \ln{N_i!}.
\end{align}
For those cells that have a galaxy subtract out the probability for zero objects.  Each of these cells thus contribute
\begin{equation}
 N_i \ln{\left(  p_i  N_{0i} \right) } - \ln{N_i!}.
\end{equation}

The exponent term is proportional to (we don't care about multiplicative terms independent of the parameters)
\begin{align}
\exp\left(- \sum_{i=1}^{N_{bin}} p_i N_{0i}  \right)
 \prod_{i=1}^{N_{bin}}  (p_i N_{0i})^{N_i} 
\end{align}

The second term is over those cells that have one galaxy:
\begin{align}
\sum_{i \in \{N_i \ne 0\}}
\ln{  \left( \sum_{\tilde{N}_i =N_i}^\infty  \mathcal{B}(N_i; \tilde{N}_i, p_i) \mathcal{P}(\tilde{N}_i ; N_{0i}(\delta))\right)} -
 \ln{  \left( \sum_{\tilde{N}_i =N_i}^\infty  \mathcal{B}(0; \tilde{N}_i, p_i) \mathcal{P}(\tilde{N}_i ; N_{0i}(\delta))\right)}\\
 =\sum_{i \in \{N_i \ne 0\}} \ln{  \left( \sum_{\tilde{N}_i =1}^\infty   \tilde{N}_i p_i (1-p_i)^{ \tilde{N}_i-1} \frac{N_{0i}^{\tilde{N}_i} e^{-N_{0i}}}{\tilde{N}_i!} \right)}
- \ln{  \left( \sum_{\tilde{N}_i =0}^\infty   (1-p_i)^{ \tilde{N}_i} \frac{N_{0i}^{\tilde{N}_i} e^{-N_{0i}}}{\tilde{N}_i!} \right)} \\
 = \sum_{i=1}^{N_{\text{galaxies}}} \ln{  \left(  p_i  {N_{0i} } \right)} \\
  = \sum_{i=1}^{N_{\text{galaxies}}} \ln{  \left(  p_i  n_{0}(x_i) \delta V \right)}
\end{align}

Then the likelihood is
\begin{align}
L(\theta; N, v) 
&\propto p(v|\theta)  \int \exp \left(- \int  p(x) n_{0}(x) d^3x\right)  \prod_{i=1}^{N_{\text{galaxies}}} p(x_i) n_0(x_i)  p(\delta| v, \theta) d\delta \\
& \approx p(v|\theta)  \int \prod_{j=1}^{N_\text{grid}} \exp \left( - \frac{p(x_j) n_{0}(x_j)}{N_\text{grid}}\right) \prod_{i=1}^{N_{\text{galaxies}}} p(x_i) n_0(x_i)  p(\delta| v, \theta) d\delta 
%&\propto \int  \exp \left(- \int  p(x) n_{0}(x) d^3x\right)  \prod_{i=1}^{N_{\text{galaxies}}} p(x_i) n_0(x_i) p(v_i| u_i) p(\delta, u | \theta) d\delta\, du \\
%& \approx  \int  \prod_{j=1}^{N_\text{grid}} \exp \left( - \frac{p(x_j) n_{0}(x_j)}{N_\text{grid}}\right) \prod_{i=1}^{N_{\text{galaxies}}} p(x_i) n_0(x_i) p(v_i| u_i) p(\delta, u | \theta) d\delta\, du.
\end{align}
The exponential term depends on $\delta$ at every differential volume.  The
remaining terms of the integrand depend only on the $\delta_i$ and $u_i$ at galaxy coordinates.

When the universe is modeled as periodic, the fields therein can be expanded as a Fourier series.  Truncating the expansion at some $k_{max}$
means that the a field can be described by a finite set of numbers.  The parameters $\delta$ and $u$ will be
the Fourier density and momentum at grid points.  Realizations of these parameters can be easily generated because their 
 covariance matrix is composed of four diagonal submatrices. 
The density and momentum at any point in real space can be constructed from the parameters using the Fourier transform.

\newpage
\section{How to analyze}
% Requires the booktabs if the memoir class is not being used
\begin{table}[htbp]
   \centering
   %\topcaption{Table captions are better up top} % requires the topcapt package
   \begin{tabular}{|c|c|} % Column formatting, @{} suppresses leading/trailing space
   	\hline
	SN Ia rest rate & $0.65 \times 2.69 \times 10^{-5} \left(\frac{h}{0.7}\right)^3 \text{Mpc}^{-3} y^{-1}$\\
	rate in $k=0.1 h \text{Mpc}^{-1}$ &$ 1.75 \times 10^{-2} \text{y}^{-1}$\\
	Full sky volume $z_{\text{max}}=0.2$ & $2.30 \times 10^{9} \left(\frac{0.7}{h}\right)^3 \text{Mpc}^3$\\
	number of cells $k=0.1 h \text{Mpc}^{-1}$   & $7.87 \times 10^5$\\
	number of SNe in full sky& $1.38 \times 10^4$ \\
	peak of matter power spectrum $P$ & $3 \times 10^4 (h^{-1} \text{Mpc})^3$\\
	$nP$ &$1.53 \mbox{y}^{-1}$ \\
	\hline
   \end{tabular}
   \caption{Useful numbers.}
   \label{tab:booktabs}
\end{table}

There are fewer point objects than cells:
\begin{itemize}
\item  After 10 years there are still fewer SNe than cells for $k=0.1 h \text{Mpc}^{-1}$. 
\item After 1 year of 100\% coverage the $gg$ survey is sample-noise dominated.  The higher number density from a
galaxy survey, which would have more points than cells, is unnecessary.
\end{itemize}

For the velocity field is makes sense to work with points.

For the density field this is not obvious.  While 

The density field probed by points requires integrals.  The numerical integration could be done by using cells though Monte Carlo integration seems to be the standard and 
is presumably more efficient.




For velocity-velocity using points is straightforward.

For density-density 

\begin{table}[htbp]
   \centering
   %\topcaption{Table captions are better up top} % requires the topcapt package
   \begin{tabular}{|c|c|c|} % Column formatting, @{} suppresses leading/trailing space
   	\hline
	Type & Pro & Con \\ \hline
	Point & \\
	Grid & \\
	Conditional Correlations Likelihood & \\
	Correlation Function Estimator & \\
	Fourier Transform & \\
	\hline
   \end{tabular}
   \caption{Methods.}
   \label{tab:booktabs}
\end{table}
\newpage
\begin{thebibliography}{9}
\bibitem{adam}
C. Adams, C. Blake. 
Improving constraints on the growth rate of structure by modelling the density-velocity cross-correlation in the 6dF Galaxy Survey. arXiv:1706.05205 [astro-ph.CO]
\bibitem[McDonald(2019)]{2019PhRvD..99d3538M} McDonald, P.\ 2019, PRD, 99, 43538
\end{thebibliography}

\end{document}~/pro