\documentclass{article}
\usepackage{amsmath}
\usepackage{amssymb}
\usepackage{bm}
\usepackage{physics}
\usepackage{verbatim}
\title{Complete draft}
%\author{Chao Ju, Alex Kim}
\begin{document}
\maketitle


There is sufficient information in the distribution of peculiar velocities and overdensities of galaxies to test general relativity on the largest scale. Suppose we have a region that encloses $N_{gal}$ number of galaxies in the universe. We can divide the region into $N_{cell}\gg 1$ number of equal volume cells whose size can be made so small that either one or no galaxy is contained in one cell. This is encoded in a vector $\textbf{N}$ of size $N_{cell}$ whose entries are either 0 or 1. For any 2 cells, we can compute their correlation, a function of the distance $r$ between the cells only. In our application, we consider $\xi_{\delta\delta}$ (the overdensity-overdensity correlation), $\xi_{\delta v}$ (overdensity-velocity correlation), $\xi_{v\delta}$ (velocity-overdensity correlation), and $\xi_{vv}$ (velocity-velocity correlation).\par

The data  is $\textbf{x}= [N_1-\bar{N}, N_2-\bar{N}, \hdots, N_{N_{cell}}-\bar{N}, \Delta v_1, \hdots, \Delta v_{N_{gal}}]^T$, where $\Delta v$ is the deviation of the measured velocity from the expected, 

We postulate that deviations from expected variance $\bar{N}$ and expected peculiar velocity $\bar{v}$ are Gaussian. Then the above correlation matrices can be used to construct a block matrix $\textbf{C}$, parameterized by $A$ and $\bm{\sigma}_v$, a vector of shot noise for peculiar velocity measurement:
\begin{align}
\textbf{C} &\equiv \textbf{C}_0 + \textbf{S}\\
%\textbf{C} &\equiv \textbf{C}_0 + A\textbf{S} \\ 
&=
\left(
\begin{array}{cc}
\bar{N}\textbf{I} & 0 \nonumber \\
0 & \bm{\sigma}_v^2\textbf{I}
\end{array}
\right)
+
\begin{pmatrix}
\bar{N}^2\xi_{gg} & \bar{N}\xi_{g v} \\
\bar{N}\xi_{v g} & \xi_{vv}
\end{pmatrix},
\end{align}
where $\xi_{gg}$, $\xi_{vv}$, and $\bar{N}\xi_{g v}$ are the galaxy overdensity and peculiar velocity correlation functions and their cross-correlation
respectively.  

The reason we have $\bar{N}$ as the variance for overdensity is that our small galaxy cells are best modeled by Poisson statistics, in which the mean equals the variance. In addition, note that the dimension of $C$ is $N_{cell}+N_{gal}$ by $N_{cell}+N_{gal}$. For the peculiar velocity covariance, we assign $\sigma^2=\infty$ to cells that do not contain any galaxy, and one can show that removing rows and columns that contain those cells has no effect but a constant offset on what we will compute later.


The science parameters $\theta$ enter through these correlation functions, and may include $\theta \in \{fD, bD, \gamma, \Omega_{M_0},\ldots\}$ and others
that describe the matter power spectrum.
For example, the parameter $fD$ enters in through the relations $\xi_{vv}\propto (fD\mu)^2$, the SN~Ia host-galaxy count overdensity
power spectrum $\xi_{g g }\propto (bD + fD\mu^2)^2$, and the galaxy-velocity cross-correlation $\xi_{vg}
\propto  (bD + fD\mu^2)fD$, where $b$ is the galaxy bias and $\mu\equiv \cos{(\hat{k} \cdot \hat{r})}$ where $\hat{r}$ is the direction of
the line of sight.  
Assuming linear dependence of the parameters,
\begin{align}
\textbf{C} 
&=
\left(
\begin{array}{cc}
\bar{N}\textbf{I} & 0 \nonumber \\
0 & \bm{\sigma}_v^2\textbf{I}
\end{array}
\right)
+ \sum_\theta \theta
\begin{pmatrix}
\bar{N}^2\xi_{gg,\theta} & \bar{N}\xi_{g v,\theta} \\
\bar{N}\xi_{v g,\theta} & \xi_{vv,\theta}
\end{pmatrix}\\
&=
\left(
\begin{array}{cc}
\bar{N}\textbf{I} & 0 \nonumber \\
0 & \bm{\sigma}_v^2\textbf{I}
\end{array}
\right)
+ \sum_\alpha \theta_\alpha \textbf{S}_\alpha .
\end{align} 

%\left(
%\begin{array}{cc}
%\bar{N}\textbf{I} & 0 \nonumber \\
%0 & \bm{\sigma}_v^2\textbf{I}
%\end{array}
%\right)
%+
%A
%\left(
%\begin{array}{cc}
%\bar{N}^2\xi_{\delta\delta} & \bar{N}\xi_{\delta v} \\
%\bar{N}\xi_{\delta v} & \xi_{vv}
%\end{array}
%\right).
%\end{align}\par


The Gaussian likelihood as a function of $\theta$
\[
\mathcal{L}(\theta) = \frac{1}{\sqrt{\left(2\pi\right)^{N_{cell}+N_{gal}} \det \textbf{C}}}\exp\left\{-\frac{1}{2}\textbf{x}^T\textbf{C}^{-1}\textbf{x}\right\}.
\]\par
To make the expression cleaner, we take the log of the likelihood and focus on the $A$-dependent part:
\begin{equation}
-2\ln\mathcal{L}(\theta) = \ln\det\textbf{C} + \textbf{x}^T\textbf{C}^{-1}\textbf{x} + \text{const}.
\label{log:eqn}
\end{equation}

The estimator and Fisher Matrix for this Gaussian likelihood are known
\begin{equation}
\hat{\theta}_\alpha = \frac{1}{2} F^{-1}_{\alpha \beta} [(\textbf{x}^T \textbf{C}^{-1} \textbf{S}_\beta \textbf{C}^{-1} \textbf{x}) - \tr(\textbf{C}^{-1} \textbf{S}_\beta \textbf{C}^{-1} \textbf{C}_0)]
\end{equation}
and
\begin{equation}
F_{\alpha\beta} = \frac{1}{2} \tr(\textbf{C}^{-1} \textbf{S}_\beta \textbf{C}^{-1} \textbf{C}_0 \textbf{S}_\beta).
\end{equation}

The term
\begin{align}
\textbf{x}^T \textbf{C}^{-1} \textbf{S}_\beta \textbf{C}^{-1} \textbf{x} & \approx \textbf{x}^T \textbf{C}_0^{-1} \textbf{S}_\beta \textbf{C}_0^{-1} \textbf{x}\\
&=(\textbf{N}-\bar{N}) \xi_{\delta\delta,\theta_\beta}  (\textbf{N}-\bar{N})  + 2  (\textbf{N}-\bar{N}) \xi_{\delta v,\theta_\beta}\frac{\mathbf{\Delta v}}{\bm{\sigma}_v^2}+ 
\frac{\mathbf{\Delta v}}{\bm{\sigma}_v^2} \xi_{vv,\theta_\beta}\frac{\mathbf{\Delta v}}{\bm{\sigma}_v^2}.
\end{align}
Noting that $N_i=1$ when the cell $i$ contains a galaxy and zero otherwise and that only cells with a peculiar velocity measurement contribute ($\sigma_v =\infty$
for cells without such a measurement), and that in the limit of infinitely small cells, $\bar{N}\sum_{i,j\in N_{cell}} = \int n dV$, where $n$ is the galaxy number density,
\begin{align*}
\textbf{x}^T \textbf{C}^{-1} \textbf{S}_\beta \textbf{C}^{-1} \textbf{x} & \approx \sum_{i,j\in N_{gal}} \xi_{\delta\delta,\theta_\beta,ij} -2 \bar{N} \sum_{i\in N_{gal},j\in N_{cell}} \xi_{\delta\delta,\theta_\beta,ij}\\
& +\bar{N}^2  \sum_{i,j\in N_{cell}}  \xi_{\delta\delta,\theta_\beta,ij} + 2 \sum_{i,j\in N_{gal}} \frac{\Delta v_j}{\sigma_{v,j}^2} \xi_{\delta v,\theta_\beta,ij} \\
& -2\bar{N}\sum_{i\in N_{cell}, j\in N_{gal}} \frac{\Delta v_j}{\sigma_{v,j}^2} \xi_{\delta v,\theta_\beta,ij}  + \sum_{i,j\in N_{gal}} \frac{\Delta v_i}{\sigma_{v,i}^2} \xi_{vv,\theta_\beta,ij} \frac{\Delta v_j}{\sigma_{v,j}^2}\\
& = \sum_{i,j\in N_{gal}} \left( \xi_{\delta\delta,\theta_\beta}(x_i,x_j) + 2  \frac{\Delta v_j}{\sigma_{v,j}^2} \xi_{\delta v,\theta_\beta}(x_i,x_j) + \frac{\Delta v_i}{\sigma_{v,i}^2} \xi_{vv,\theta_\beta}(x_i,x_j) \frac{\Delta v_j}{\sigma_{v,j}^2}\right)\\
&-2  \sum_{i\in N_{gal}}\int n(x_i) \left(\xi_{\delta\delta,\theta_\beta}(x_i,y) + \frac{\Delta v_i}{\sigma_{v,i}^2} \xi_{v \delta,\theta_\beta}(x_i,y) \right) d^3y\\
&  + \int n(x) n(y) \xi_{\delta\delta,\theta_\beta}(x,y)d^3xd^3y.
\end{align*}

The trace term depends on only the zero-lag correlation
\begin{align}
 \tr(\textbf{C}^{-1} \textbf{S}_\beta \textbf{C}^{-1} \textbf{C}_0) & \approx  \tr(\textbf{C}_0^{-1} \textbf{S}_\beta) \\
 & = \bar{N} \tr(\xi_{\delta\delta,\beta}) + \tr(\frac{1}{\sigma^2_v} \xi_{vv,\beta})\\
 & = \int n(x) \xi_{\delta \delta,\beta}(x,x) d^3x +  \sum_{i\in N_{gal}} \frac{1}{\sigma^2_{v,i}} \xi_{vv,\beta}(x_i,x_i).
\end{align}

The term in the Fisher matrix, to the leading term in $\bar{N}$ is
\begin{align}
 \tr(\textbf{C}^{-1} \textbf{S}_\alpha \textbf{C}^{-1} \textbf{S}_\beta) & \approx  \tr(\textbf{C}_0^{-1} \textbf{S}_\alpha \textbf{C}_0^{-1} \textbf{S}_\beta)  \\
 & = \bar{N}^2 \tr(\xi_{\delta \delta,\alpha}\xi_{\delta \delta,\beta})+
 \bar{N} \tr(\xi_{\delta v,\alpha} \frac{1}{\bm{\sigma}^2_v} \xi_{v\delta,\beta} ) + \tr(\frac{1}{\bm{\sigma}^2_v} \xi_{vv,\alpha}  \frac{1}{\bm{\sigma}^2_v}  \xi_{vv,\beta})\\
 & = \int n(x) n(y) \xi_{\delta \delta,\alpha}(x,y)\xi_{\delta \delta,\beta}(y,x) d^3x d^3y \\
 & + \int n(x) \sum_{i\in N_{gal}} \xi_{\delta v,\alpha}(x,x_i) \frac{1}{\sigma^2_{v,i}} \xi_{v\delta,\beta} (x_i,x) d^3x \\
 & +  \sum_{i,j\in N_{gal}} \frac{1}{\sigma^2_{v,i}} \xi_{vv,\alpha}(x_i,x_j)  \frac{1}{\sigma^2_{v,j}}  \xi_{vv,\beta}(x_j,x_i).
\end{align}

%  ---- Change
%NEW STUFF
\begin{comment}
We now expand the log-likelihood as a Taylor series for small $\theta$ around $\theta=0$.

First, we focus on the  term $\textbf{x}^T\textbf{C}^{-1}\textbf{x}$. We rewrite $\textbf{C}$ by normalizing out its uncorrelated $\textbf{C}_0$ term
by   introducing the matrices $\textbf{M}$ and $\textbf{X}$,
where $\textbf{M}=\sqrt{\textbf{C}_0}$ and $\textbf{S} = \textbf{M} \textbf{X} \textbf{M}$ such that $\textbf{X}=\textbf{M}^{-1}\textbf{S}\textbf{M}^{-1}$. Now, our matrix $\textbf{C}$ can be re-written in terms of $\textbf{X}$ and $\textbf{M}$:
\begin{equation}
\textbf{C} = \textbf{M}(\textbf{I}+A\textbf{X})\textbf{M}.
\end{equation}
Expanded as a Taylor series
\begin{align}
\textbf{C}^{-1} & = \textbf{M}^{-1}(1-A\textbf{X}+A^2 \textbf{X}^2 + \ldots) \textbf{M}^{-1} \\
& = \textbf{C}_0^{-1}   -A \textbf{C}_0^{-1}\textbf{S}\textbf{C}_0^{-1} +\frac{A^2}{2} \textbf{C}_0^{-1}\textbf{S}\textbf{C}_0^{-1}\textbf{S}\textbf{C}_0^{-1} +O(A^3) + \ldots 
\end{align}
%= -A \textbf{C}_0^{-1}\textbf{S}\textbf{C}_0^{-1} + \text{const}.
%\] \par


A term of interest is 
\begin{align}
\textbf{x}^T \textbf{C}_0^{-1}\textbf{S}\textbf{C}_0^{-1} \textbf{x}& = (\textbf{N}-\bar{N}) \xi_{\delta\delta}  (\textbf{N}-\bar{N})  + 2  (\textbf{N}-\bar{N}) \xi_{\delta v}\frac{\mathbf{\Delta v}}{\bm{\sigma}_v^2}+ 
\frac{\mathbf{\Delta v}}{\bm{\sigma}_v^2} \xi_{vv}\frac{\mathbf{\Delta v}}{\bm{\sigma}_v^2}.
\end{align}
Noting that $N_i=1$ when the cell $i$ contains a galaxy and zero otherwise and that only cells with a peculiar velocity measurement contribute ($\sigma_v =\infty$
for cells without such a measurement), and that in the limit of infinitely small cells, $\bar{N}\sum_{i,j\in N_{cell}} = n \int dV$, where $n$ is the galaxy number density,
\begin{align*}
\textbf{x}^T \textbf{C}_0^{-1}\textbf{S}\textbf{C}_0^{-1} \textbf{x}& = \sum_{i,j\in N_{gal}} \xi_{\delta\delta,ij} -2 \bar{N} \sum_{i\in N_{gal},j\in N_{cell}} \xi_{\delta\delta,ij}\\
& +\bar{N}^2  \sum_{i,j\in N_{cell}}  \xi_{\delta\delta,ij} + 2 \sum_{i,j\in N_{gal}} \frac{\Delta v_j}{\sigma_{v,j}^2} \xi_{\delta v,ij} \\
& -2\bar{N}\sum_{i\in N_{cell}, j\in N_{gal}} \frac{\Delta v_j}{\sigma_{v,j}^2} \xi_{\delta v,ij}  + \sum_{i,j\in N_{gal}} \frac{\Delta v_i}{\sigma_{v,i}^2} \xi_{vv,ij} \frac{\Delta v_j}{\sigma_{v,j}^2}\\
& = \sum_{i,j\in N_{gal}} \left( \xi_{\delta\delta,ij} + 2  \frac{\Delta v_j}{\sigma_{v,j}^2} \xi_{\delta v,ij} + \frac{\Delta v_i}{\sigma_{v,i}^2} \xi_{vv,ij} \frac{\Delta v_j}{\sigma_{v,j}^2}\right)\\
&-2 n \sum_{i\in N_{gal}}\int \left(\xi_{\delta\delta,iV} + \frac{\Delta v_i}{\sigma_{v,i}^2} \xi_{v \delta,iV}\right) dV\\
& +n^2  \int \xi_{\delta\delta,V_1V_2}dV_1dV_2.
\end{align*}

\end{comment}

We evaluate $\textbf{x}^T \textbf{C}^{-1}\textbf{x}$ by splitting up the counts and velocities.  The data vector $\textbf{x}$ is composed of the counts
 $\textbf{N}-\bar{N}$ and peculiar velocities $\mathbf{\Delta v}$, whereas $\textbf{S}$ is comprised of $\bar{N}^2\xi_{\delta\delta}$, $\bar{N}\xi_{\delta v}$,
and $\xi_{vv}$.  In this manner
\begin{align}
\textbf{x}^T \textbf{C}_0^{-1}\textbf{S}\textbf{C}_0^{-1} \textbf{x}& = (\textbf{N}-\bar{N}) \xi_{\delta\delta}  (\textbf{N}-\bar{N})  + 2  (\textbf{N}-\bar{N}) \xi_{\delta v}\frac{\mathbf{\Delta v}}{\bm{\sigma}_v^2}+ 
\frac{\mathbf{\Delta v}}{\bm{\sigma}_v^2} \xi_{vv}\frac{\mathbf{\Delta v}}{\bm{\sigma}_v^2}.
\end{align}
Noting that $N_i=1$ when the cell $i$ contains a galaxy and zero otherwise and that only cells with a peculiar velocity measurement contribute ($\sigma_v =\infty$
for cells without such a measurement), and that in the limit of infinitely small cells, $\bar{N}\sum_{i,j\in N_{cell}} = n \int dV$, where $n$ is the galaxy number density,
\begin{align*}
\textbf{x}^T \textbf{C}_0^{-1}\textbf{S}\textbf{C}_0^{-1} \textbf{x}& = \sum_{i,j\in N_{gal}} \xi_{\delta\delta,ij} -2 \bar{N} \sum_{i\in N_{gal},j\in N_{cell}} \xi_{\delta\delta,ij}\\
& +\bar{N}^2  \sum_{i,j\in N_{cell}}  \xi_{\delta\delta,ij} + 2 \sum_{i,j\in N_{gal}} \frac{\Delta v_j}{\sigma_{v,j}^2} \xi_{\delta v,ij} \\
& -2\bar{N}\sum_{i\in N_{cell}, j\in N_{gal}} \frac{\Delta v_j}{\sigma_{v,j}^2} \xi_{\delta v,ij}  + \sum_{i,j\in N_{gal}} \frac{\Delta v_i}{\sigma_{v,i}^2} \xi_{vv,ij} \frac{\Delta v_j}{\sigma_{v,j}^2}\\
& = \sum_{i,j\in N_{gal}} \left( \xi_{\delta\delta,ij} + 2  \frac{\Delta v_j}{\sigma_{v,j}^2} \xi_{\delta v,ij} + \frac{\Delta v_i}{\sigma_{v,i}^2} \xi_{vv,ij} \frac{\Delta v_j}{\sigma_{v,j}^2}\right)\\
&-2 n \sum_{i\in N_{gal}}\int \left(\xi_{\delta\delta,iV} + \frac{\Delta v_i}{\sigma_{v,i}^2} \xi_{v \delta,iV}\right) dV\\
& +n^2  \int \xi_{\delta\delta,V_1V_2}dV_1dV_2.
\end{align*}
\begin{comment}

Now consider the next term $\textbf{C}_0^{-1}\textbf{S}\textbf{C}_0^{-1}\textbf{S}\textbf{C}_0^{-1}$.  The middle section is
\begin{align}
\textbf{S}\textbf{C}_0^{-1}\textbf{S} & =
\begin{pmatrix}
 \bar{N}^2  \xi_{\delta v}\frac{1}{\bm{\sigma}_v^2}  \xi_{v \delta}  & \bar{N} \xi_{\delta v} \frac{1} {\bm{\sigma}_v^2} \xi_{vv}  \\
 \bar{N} \xi_{v v} \frac{1} {\bm{\sigma}_v^2} \xi_{v \delta}   &  \xi_{vv} \frac{1}{\bm{\sigma}_v^2} \xi_{vv}
\end{pmatrix}
\end{align}
to leading order of $\bar{N}$ anticipating that we will take the limit where it goes to zero.
Then
\begin{align}
\textbf{x}^T \textbf{C}_0^{-1}\textbf{S}\textbf{C}_0^{-1}\textbf{S}\textbf{C}_0^{-1} \textbf{x}
& = \sum_{i,j\in N_{gal}} \left( \left(\xi_{\delta v}\frac{1}{\bm{\sigma}_v^2}  \xi_{v \delta}\right)_{ij}
 + 2  \frac{\Delta v_j}{\sigma_{v,j}^2} \left(\xi_{\delta v} \frac{1} {\bm{\sigma}_v^2} \xi_{vv}  \right)_{ij} + \frac{\Delta v_i}{\sigma_{v,i}^2} \left( \xi_{vv} \frac{1}{\bm{\sigma}_v^2} \xi_{vv}\right)_{ij} \frac{\Delta v_j}{\sigma_{v,j}^2}\right)\\
&-2 n \sum_{i\in N_{gal}}\int \left( \left(\xi_{\delta v}\frac{1}{\bm{\sigma}_v^2}  \xi_{v \delta}\right)_{iV} + \frac{\Delta v_i}{\sigma_{v,i}^2} \left( \xi_{v v} \frac{1} {\bm{\sigma}_v^2} \xi_{v \delta}  \right)_{iV} \right) dV\\
& +n^2  \int  \left(\xi_{\delta v}\frac{1}{\bm{\sigma}_v^2}  \xi_{v \delta}\right)_{V_1V_2}dV_1dV_2.
\end{align}

We have assembled the contributions of $\textbf{x}^T \textbf{C}^{-1}\textbf{x}$ to second order in $A$.

Returning to the other contribution to the log-likelihood in Eq.~\ref{log:eqn} we have
\begin{align*}
\ln\det(\textbf{C}) &=\ln\det(\textbf{M}(\textbf{I}+A\textbf{X})\textbf{M})) \\
&= \ln\det(\textbf{I}+A\textbf{X}) +\text{const} \\
&= A \tr \textbf{X} -\frac{A^2}{2}\tr \textbf{X}^2+ O(A^3) + \ldots\\
\end{align*}
Now
\begin{equation}
\textbf{X} = \textbf{M}^{-1}\textbf{S}\textbf{M}^{-1} = 
\begin{pmatrix}
\bar{N}\xi_{\delta\delta} & \frac{\sqrt{\bar{N}}}{\sigma_{v}} \xi_{\delta v} \\
\frac{\sqrt{\bar{N}}}{\sigma_{v}} \xi_{v\delta} & \frac{1}{\sigma_{v}^2} \xi_{vv} 
\end{pmatrix}.
\end{equation}
The contribution to the trace of the upper diagonal in the limit of small cell volumes is
\begin{equation}
\bar{N} \tr \xi_{\delta\delta} = n \int \xi_{\delta\delta,VV} dV
\end{equation}
and the lower diagonal is
\begin{equation}
\tr  \frac{1}{\sigma_{v}^2} \xi_{vv} =  \sum_{i \in N_{gal}}   \frac{1}{\sigma_{v,i}^2} \xi_{vv,ii}.
\end{equation}

The trace of $\textbf{X}^2$ has terms, to leading order in $\bar{N}$, has terms
\begin{equation}
\bar{N}  \tr  \xi_{\delta v} \frac{1}{\sigma_{v}}  \frac{1}{\sigma_{v}}  \xi_{v \delta} = n \int \left( \xi_{\delta v} \frac{1}{\sigma_{v}}  \frac{1}{\sigma_{v}}  \xi_{v \delta} \right)_{VV} dV
\end{equation}
and 
\begin{equation}
\tr   \xi_{vv} \frac{1}{\sigma_{v}^2}  \frac{1}{\sigma_{v}^2} \xi_{vv} =  \sum_{i \in N_{gal}} \left(  \xi_{vv} \frac{1}{\sigma_{v}^2}  \frac{1}{\sigma_{v}^2} \xi_{vv} \right)_{ii}
\end{equation}

Putting everything together, the expansion of the log-likelihood in $A$ has the first order coefficient
\begin{align}
c_1 &=- \sum_{i,j\in N_{gal}} \left( \xi_{\delta\delta,ij} + 2  \frac{\Delta v_j}{\sigma_{v,j}^2} \xi_{\delta v,ij} + \frac{\Delta v_i}{\sigma_{v,i}^2} \xi_{vv,ij} \frac{\Delta v_j}{\sigma_{v,j}^2}\right)\\
& +\sum_{i \in N_{gal}} \left( \frac{1}{\sigma_{v,i}^2} \xi_{vv,ii}   + 2n \int \left(\xi_{\delta\delta,iV} + \frac{\Delta v_i}{\sigma_{v,i}^2} \xi_{v \delta,iV}\right) dV  \right)\\
& +n \int \xi_{\delta\delta,VV} dV  -n^2  \int \xi_{\delta\delta,V_1V_2}dV_1dV_2
\end{align}
and the second order coefficient
\begin{align}
c_2 & = \frac{1}{2} \Biggl( 
 \sum_{i,j\in N_{gal}} \left( \left(\xi_{\delta v}\frac{1}{\bm{\sigma}_v^2}  \xi_{v \delta}\right)_{ij}
 + 2  \frac{\Delta v_j}{\sigma_{v,j}^2} \left(\xi_{\delta v} \frac{1} {\bm{\sigma}_v^2} \xi_{vv}  \right)_{ij} + \frac{\Delta v_i}{\sigma_{v,i}^2} \left( \xi_{vv} \frac{1}{\bm{\sigma}_v^2} \xi_{vv}\right)_{ij} \frac{\Delta v_j}{\sigma_{v,j}^2}\right)\\
&- \sum_{i \in N_{gal}}\left( \left(  \xi_{vv} \frac{1}{\sigma_{v}^2}  \frac{1}{\sigma_{v}^2} \xi_{vv} \right)_{ii}  +2n \int \left( \left(\xi_{\delta v}\frac{1}{\bm{\sigma}_v^2}  \xi_{v \delta}\right)_{iV} + \frac{\Delta v_i}{\sigma_{v,i}^2} \left( \xi_{v v} \frac{1} {\bm{\sigma}_v^2} \xi_{v \delta}  \right)_{iV} \right) dV  \right)\\
&-n  \int \left( \xi_{\delta v} \frac{1}{\sigma_{v}}  \frac{1}{\sigma_{v}}  \xi_{v \delta} \right)_{VV} dV  +n^2  \int  \left(\xi_{\delta v}\frac{1}{\bm{\sigma}_v^2}  \xi_{v \delta}\right)_{V_1V_2}dV_1dV_2 \Biggr).
\end{align}

In this second-order approximation of the log-likelihood, the minimum occurs at $A_{\text{min}}=-c_1/2c_2$ and the second derivative is $2c_2$.


END NEW STUFF

Given the enormous size of $\textbf{C}$, computing the log likelihood directly is not possible. In the following, we seek an approximation method accurate to first order in $A$. \par
First, we focus on the second term, $\textbf{x}^T\textbf{C}^{-1}\textbf{x}$. Looking at the definition of $\textbf{C}$ in equation (1), we want to factor $\textbf{C}_0$ into an identity matrix. To do so, we introduce matrix $\textbf{M}$ and $\textbf{X}$. We let $\textbf{M}=\sqrt{\textbf{C}_0}$, and $\textbf{S} = \textbf{M} \textbf{X} \textbf{M}$. Then $\textbf{X}=\textbf{M}^{-1}\textbf{S}\textbf{M}^{-1}$. Now, our matrix $\textbf{C}$ can be re-written in terms of $\textbf{X}$ and $\textbf{M}$:
\begin{equation}
\textbf{C} = \textbf{M}(\textbf{I}+A\textbf{X})\textbf{M},
\end{equation}
and its inverse, to first order, is
\[
\textbf{C}^{-1} = \textbf{M}^{-1}(1-A\textbf{X}) \textbf{M}^{-1} = -A \textbf{C}_0^{-1}\textbf{S}\textbf{C}_0^{-1} + \text{const}.
\] \par
Having this expression, we can now compute the second term in equation (2). Denoting the value of this term by $s$, we have
\begin{align*}
s &=\textbf{x}^T \textbf{C}^{-1}\textbf{x} \\
&= -A\left(\textbf{x}^T \textbf{C}_0^{-1}\right)\textbf{S}\left(\textbf{C}_0^{-1}\textbf{x}\right).
\end{align*} \par
Using the equations for covariance calculation from \cite{adam}, we can break the above expression into a sum of 6 terms and write out the explicit solution for each (apart from a zeroth-order constant). Let 
\[
-\left(\textbf{x}^T \textbf{C}_0^{-1}\right)\textbf{S}\left(\textbf{C}_0^{-1}\textbf{x}\right) = s_1 +s_2+\hdots +s_6
\]
\begin{align*}
s_1&= -\sum_{i,j\in N_{gal}} \xi_{\delta\delta,ij},  \\
s_2 &= 2 \sum_{i\in N_{gal},j\in N_{cell}} \bar{N} \xi_{\delta\delta,ij}, \\
&= 2\bar{N} \frac{b^2}{2\pi} \sum_{i\in N_{gal}} \int r^2drd\Omega  \int dk P(k) k^2 j_0 (k |\textbf{r}-\textbf{r}_i|), \\
s_3 &= -\sum_{i,j\in N_{cell}} \bar{N}^2  \xi_{\delta\delta,ij}, \\ 
&=-\bar{N}^2 \frac{b^2}{2\pi} \int r_1^2dr_1d\Omega_1 \int r_2^2dr_2 d\Omega_2 \int dk P(k) k^2 j_0 (k |\textbf{r}_1-\textbf{r}_2|), \\
s_4 &=- \sum_{i,j\in N_{gal}} \frac{\Delta v_i}{\sigma_{v,i}^2} \xi_{vv,ij} \frac{\Delta v_j}{\sigma_{v,j}^2}, \\
s_5 &= -2 \sum_{i,j\in N_{gal}} \frac{\Delta v_j}{\sigma_{v,j}^2} \xi_{\delta v,ij}, \\
s_6 &= 2\sum_{i\in N_{cell}, j\in N_{gal}} \bar{N}\frac{\Delta v_j}{\sigma_{v,j}^2} \xi_{\delta v,ij},  \\
&= 2\bar{N}\sum_{j\in N_{gal}} \frac{aHfb\Delta v_j}{2\pi^2 \sigma^2_{v,j}}\int r^2 dr d\Omega \int dk P(k) k ((\hat{\textbf{r}}-\hat{\textbf{r}_j})\cdot\hat{\textbf{r}_j}) j_1(k|\textbf{r}-\textbf{r}_j|). 
%s_7 &= A  \sum_{j\in N_{cell}, i\in N_{gal}} \bar{N}\frac{\Delta v_i}{\sigma_{v,i}^2} \xi_{v\delta,ij}  \\
%&= A\bar{N} \sum_{i\in N_{gal}} \frac{+aHfb\Delta v_i}{2\pi^2 \sigma^2_{v,i}}\int r^2 dr d\Omega \int dk P(k) k (\textbf{r}\cdot\textbf{r}_i) j_1(k|\textbf{r}-\textbf{r}_i|)
\end{align*} \par
We note that in the above expressions, $P(k)$ is the power spectrum, $a$ the expansion factor, $H$ the Hubble constant, $b$ the galaxy bias, $f$ the growth rate of structure, and $j_l$ the spherical Bessel function. \par
Having computed the second term in equation (2), we now turn to the first term, $\ln\det\textbf{C}$. Using equation (3) to factor $\textbf{C}$, we have
\begin{align*}
\ln\det(\textbf{C}) &=\ln\det(\textbf{M}(\textbf{I}+A\textbf{X})\textbf{M})) \\
&= \ln\det(\textbf{I}+A\textbf{X}) +\text{const} \\
&\approx A \tr \textbf{X} \\
&= A\tr(\textbf{M}^{-1}\textbf{S}\textbf{M}^{-1}) \\
&= A\tr 
\left(
\begin{array}{cc}
\bar{N}\xi_{\delta\delta} & \frac{\sqrt{\bar{N}}}{\sigma_{v}} \xi_{\delta v} \\
\frac{\sqrt{\bar{N}}}{\sigma_{v}} \xi_{v\delta} & \frac{1}{\sigma_{v}^2} \xi_{vv} 
\end{array}
\right),
\end{align*}
The trace can be broken up into 2 sums, $d_1+d_2$:
\begin{align*}
d_1 = \frac{b^2\bar{N}}{2\pi} \int r^2drd\Omega  \int dk P(k) k^2 
\end{align*}
since $j_0(0)=1$.
\[
d_2 = \sum_{i}\xi_{vv,ii}\sigma_i^{-2}
\] \par
In the limit as $N_{cell}$ goes to infinity, $\bar{N}\propto 1/N_{cell}$ goes to zero. Therefore, terms that contain $k$ factors of $N_{cell}$ will vanish in the $N_{cell}\to\infty$ limit unless they also contain $k$ summations over $N_{cell}$: $\sum_{N_{cell}}$. The reason is that $\sum_{N_{cell}}\bar{N}$ stays finite when $N_{cell}\to \infty$. Looking at the above $s$ and $d$ terms, we see that all of them are finite in the large $\bar{N}$ limit, and the the full likelihood to first order in $\bar{N}$ is
\[
\boxed{A(s_1+s_2+s_4+s_5+s_6+d_1+d_2)}
\]\par
$s_3$ is not selected because it is quadratic in $\bar{N}$.
\section{Inversion to second order}
The $O(A^2)$ prefactor in $\textbf{C}^{-1}$ is 
\begin{align*}
\textbf{M}^{-1}\textbf{X}^2\textbf{M}^{-1} &= \textbf{M}^{-1} \textbf{M}^{-1} \textbf{S} \textbf{M}^{-1}\textbf{M}^{-1} \textbf{S} \textbf{M}^{-1}\textbf{M}^{-1} \\
&=\textbf{C}_0^{-1}\textbf{S}\textbf{C}_0^{-1}\textbf{S}\textbf{C}_0^{-1}
\end{align*}\par
I first sight I wanted to move $\textbf{S}$ across the $\textbf{C}_0^{-1}$ by using a commutator, thinking that it might simplify things a lot. But it turned out that the commutator did not lead to the expected cancelation, so let's do it the hard way. We first do $\textbf{C}_0^{-1}\textbf{S}\textbf{C}_0^{-1}\textbf{S}$, henceforth denoting the inverse of $\bm{\sigma}^2_v\textbf{I}$ as $\bm{\sigma}^{-1}$:
\begin{align*}
\textbf{C}_0^{-1}\textbf{S}\textbf{C}_0^{-1}\textbf{S} &=
\left(
\begin{array}{cc}
\bar{N}\xi_{\delta\delta} & \xi_{\delta v} \\
\bar{N}\bm{\sigma}^{-1} \xi_{v\delta} & \bm{\sigma}^{-1} \xi_{vv}
\end{array}
\right)
\left(
\begin{array}{cc}
\bar{N}\xi_{\delta\delta} & \xi_{\delta v} \\
\bar{N}\bm{\sigma}^{-1} \xi_{v\delta} & \bm{\sigma}^{-1} \xi_{vv}
\end{array}
\right) \\
&=
\left(
\begin{array}{cc}
\bar{N}^2 \xi_{\delta\delta}^2 + \bar{N}\xi_{\delta v}\bm{\sigma}^{-1} \xi_{v\delta} & \bar{N}\xi_{\delta\delta}\xi_{\delta v}+\xi_{\delta v}\bm{\sigma}^{-1}\xi_{vv} \\
\bar{N}^2\bm{\sigma}^{-1} \xi_{v\delta}\xi_{\delta\delta}+\bar{N}\bm{\sigma}^{-1} \xi_{vv}\bm{\sigma}^{-1} \xi_{v\delta} & \bar{N}\bm{\sigma}^{-1} \xi_{v\delta}\xi_{\delta v} + \bm{\sigma}^{-1} \xi_{vv}\bm{\sigma}^{-1} \xi_{vv}
\end{array}
\right)
\end{align*}\par
Next, we multiply $\textbf{C}_0^{-1}$ from the right and get
\begin{align*}
\textbf{C}_0^{-1}\textbf{S}\textbf{C}_0^{-1}\textbf{S}\textbf{C}_0^{-1} &=
\left(
\begin{array}{cc}
\bar{N}^2 \xi_{\delta\delta}^2 + \bar{N}\xi_{\delta v}\bm{\sigma}^{-1} \xi_{v\delta} & \bar{N}\xi_{\delta\delta}\xi_{\delta v}+\xi_{\delta v}\bm{\sigma}^{-1}\xi_{vv} \\
\bar{N}^2\bm{\sigma}^{-1} \xi_{v\delta}\xi_{\delta\delta}+\bar{N}\bm{\sigma}^{-1} \xi_{vv}\bm{\sigma}^{-1} \xi_{v\delta} & \bar{N}\bm{\sigma}^{-1} \xi_{v\delta}\xi_{\delta v} + \bm{\sigma}^{-1} \xi_{vv}\bm{\sigma}^{-1} \xi_{vv}
\end{array}
\right)
\left(
\begin{array}{cc}
\bar{N}^{-1}\textbf{I} & 0 \\
0 & \bm{\sigma}^{-1}
\end{array}
\right) \\
&=
\left(
\begin{array}{cc}
\bar{N} \xi_{\delta\delta}^2 + \xi_{\delta v}\bm{\sigma}^{-1} \xi_{v\delta} & \bar{N}\xi_{\delta\delta}\xi_{\delta v}\bm{\sigma}^{-1}+\xi_{\delta v}\bm{\sigma}^{-1}\xi_{vv}\bm{\sigma}^{-1} \\
\bar{N}\bm{\sigma}^{-1} \xi_{v\delta}\xi_{\delta\delta}+\bm{\sigma}^{-1} \xi_{vv}\bm{\sigma}^{-1} \xi_{v\delta} & \bar{N}\bm{\sigma}^{-1} \xi_{v\delta}\xi_{\delta v}\bm{\sigma}^{-1} + \bm{\sigma}^{-1} \xi_{vv}\bm{\sigma}^{-1} \xi_{vv}\bm{\sigma}^{-1}
\end{array}
\right)
\end{align*}\par
Let's denote the 4 blocks in the above matrix by
\[
\textbf{C}_0^{-1}\textbf{S}\textbf{C}_0^{-1}\textbf{S}\textbf{C}_0^{-1}=
\left(
\begin{array}{cc}
\textbf{A} & \textbf{B} \\
\textbf{C} & \textbf{D}
\end{array}
\right)
\]\par
Let's now compute each block (the summations below are taken from $0$ to $N_{gal}$ unless otherwise specified). 
\begin{align*}
\textbf{A}_{ij} &=(\bar{N} \xi_{\delta\delta}^2 + \xi_{\delta v}\bm{\sigma}^{-1} \xi_{v\delta})_{ij} \\
&= \frac{\bar{N}b^4}{4\pi^4} \int r^2drd\Omega \iint dk_1dk_2 P(k_1)P(k_2) k_1^2k_2^2j_0(k_1|\textbf{r}-\textbf{r}_i|)j_0(k_2|\textbf{r}-\textbf{r}_j|) +\sum_{m}\sigma_{m}^{-2}\xi_{\delta v,im}\xi_{v\delta,mj} \\
&=\frac{\bar{N}b^4}{4\pi^4} \int r^2drd\Omega \iint dk_1dk_2 P(k_1)P(k_2) k_1^2k_2^2j_0(k_1|\textbf{r}-\textbf{r}_i|)j_0(k_2|\textbf{r}-\textbf{r}_j|)\\
&+\frac{a^2H^2f^2b^2}{4\pi^2} \sum_m \sigma_{m}^{-2} \iint dk_1dk_2 P(k_1)P(k_2) k_1k_2j_1(k_1|\textbf{r}_i-\textbf{r}_m|)j_1(k_2|\textbf{r}_m-\textbf{r}_j|)(\hat{\textbf{r}}_i-\hat{\textbf{r}}_m)\cdot\hat{\textbf{r}}_m(\hat{\textbf{r}}_m-\hat{\textbf{r}}_j)\cdot\hat{\textbf{r}}_m\\
\textbf{B}_{ij} &= \frac{\bar{N}aHfb^3\sigma_j^{-2}}{4\pi^4}\int r^2drd\Omega \iint dk_1dk_2 P(k_1)P(k_2)k_1^2k_2(\hat{\textbf{r}}-\hat{\textbf{r}}_j)\cdot\hat{\textbf{r}}_jj_0(k_1|\textbf{r}-\textbf{r}_i|)j_1(k_2|\textbf{r}-\textbf{r}_j|)\\
&+ \sigma_j^{-2}\sum_m \sigma_m^{-2}\xi_{\delta v,im}\xi_{vv,mj} \\
&=\frac{\bar{N}aHfb^3\sigma_j^{-2}}{4\pi^4}\int r^2drd\Omega \iint dk_1dk_2 P(k_1)P(k_2)k_1^2k_2(\hat{\textbf{r}}-\hat{\textbf{r}}_j)\cdot\hat{\textbf{r}}_jj_0(k_1|\textbf{r}-\textbf{r}_i|)j_1(k_2|\textbf{r}-\textbf{r}_j|) \\
&+\frac{\sigma_j^{-2}aHfb}{2\pi^2}\sum_m \sigma_m^{-2} \xi_{vv,mj}\int dk P(k)k j_1(k|\textbf{r}_i-\textbf{r}_m|)(\hat{\textbf{r}}_i-\hat{\textbf{r}}_m)\cdot\hat{\textbf{r}}_m \\
\textbf{C}_{ij} &=\frac{\bar{N}aHfb^3\sigma_i^{-2}}{4\pi^4}\int r^2drd\Omega \iint dk_1dk_2 P(k_1)P(k_2)k_1k_2^2(\hat{\textbf{r}_i}-\hat{\textbf{r}})\cdot\hat{\textbf{r}}_ij_0(k_2|\textbf{r}-\textbf{r}_j|)j_1(k_1|\textbf{r}-\textbf{r}_i|)\\
&+ \sigma_j^{-2}\sum_m \sigma_m^{-2}\xi_{vv,im}\xi_{v\delta,mj} \\
&=\frac{\bar{N}aHfb^3\sigma_i^{-2}}{4\pi^4}\int r^2drd\Omega \iint dk_1dk_2 P(k_1)P(k_2)k_1k_2^2(\hat{\textbf{r}_i}-\hat{\textbf{r}})\cdot\hat{\textbf{r}}_ij_0(k_2|\textbf{r}-\textbf{r}_j|)j_1(k_1|\textbf{r}-\textbf{r}_i|)\\
&+\frac{\sigma_j^{-2}aHfb}{2\pi^2}\sum_m \sigma_m^{-2} \xi_{vv,im}\int dk P(k)k j_1(k|\textbf{r}_m-\textbf{r}_j|)(\hat{\textbf{r}}_m-\hat{\textbf{r}}_j)\cdot\hat{\textbf{r}}_m\\
\textbf{D}_{ij} &=\frac{\bar{N}a^2H^2f^2b^2\sigma_i^{-2}\sigma_j^{-2}}{4\pi^4} \int r^2drd\Omega \iint dk_1dk_2 P(k_1)P(k_2)k_1^2k_2^2(\hat{\textbf{r}_i}-\hat{\textbf{r}})\cdot\hat{\textbf{r}}_i(\hat{\textbf{r}}-\hat{\textbf{r}}_j)\cdot\hat{\textbf{r}}_j\times \\ 
&\times j_1(k_1|\textbf{r}-\textbf{r}_i|)j_1(k_2|\textbf{r}-\textbf{r}_j|) +\sigma_i^{-2}\sigma_j^{-2}\sum_m \sigma_m^{-2}\xi_{vv,im}\xi_{vv,mj}
\end{align*}\par
Having obtained the 4 block matrices, we now proceed with the calculation of $\textbf{x}^T\textbf{C}_0^{-1}\textbf{S}\textbf{C}_0^{-1}\textbf{S}\textbf{C}_0^{-1}\textbf{x}$. Similar to what we did for the first order, we break the sum into the following terms (again, summation is within the range of $N_{gal}$ by default unless otherwise stated):
\begin{align*}
S_1 &= \sum_{i, j} \textbf{A}_{ij} \\
&=\frac{\bar{N}b^4}{4\pi^4}\sum_{i, j} \int r^2drd\Omega \iint dk_1dk_2 P(k_1)P(k_2) k_1^2k_2^2j_0(k_1|\textbf{r}-\textbf{r}_i|)j_0(k_2|\textbf{r}-\textbf{r}_j|)  \\
&+ \frac{a^2H^2f^2b^2}{4\pi^2} \sum_{m,i,j} \sigma_{m}^{-2} \iint dk_1dk_2 P(k_1)P(k_2) k_1k_2j_1(k_1|\textbf{r}_i-\textbf{r}_m|)j_1(k_2|\textbf{r}_m-\textbf{r}_j|)(\hat{\textbf{r}}_i-\hat{\textbf{r}}_m)\cdot\hat{\textbf{r}}_m(\hat{\textbf{r}}_m-\hat{\textbf{r}}_j)\cdot\hat{\textbf{r}}_m\\
S_2 &=-2\bar{N} \sum_{i\in N_{gal},j\in N_{cell}} \textbf{A}_{ij}\\
&=\frac{-\bar{N}^2 b^4}{2\pi^4}\sum_{i}\int r^2r_j^2drd\Omega dr_j d\Omega_j \iint dk_1dk_2 P(k_1)P(k_2) k_1^2k_2^2j_0(k_1|\textbf{r}-\textbf{r}_i|)j_0(k_2|\textbf{r}-\textbf{r}_j|)-\frac{a^2H^2f^2b^2\bar{N}}{2\pi^2}\times \\
&\times  \sum_{m,i} \sigma_{m}^{-2}\int r_j^2dr_j d\Omega_j \iint dk_1dk_2 P(k_1)P(k_2) k_1k_2j_1(k_1|\textbf{r}_i-\textbf{r}_m|)j_1(k_2|\textbf{r}_m-\textbf{r}_j|)(\hat{\textbf{r}}_i-\hat{\textbf{r}}_m)\cdot\hat{\textbf{r}}_m \times \\
&\times (\hat{\textbf{r}}_m-\hat{\textbf{r}}_j)\cdot\hat{\textbf{r}}_m \\
S_3 &= \bar{N}^2 \sum_{i, j\in N_{cell}} \textbf{A}_{ij} \\
&=\frac{\bar{N}^3b^4}{4\pi^4}\int r^2r_i^2r_j^2 drd\Omega d\Omega_i d\Omega_j \iint dk_1dk_2 P(k_1)P(k_2) k_1^2k_2^2j_0(k_1|\textbf{r}-\textbf{r}_i|)j_0(k_2|\textbf{r}-\textbf{r}_j|)  \\
&+ \frac{a^2H^2f^2b^2 \bar{N}^2}{4\pi^2} \sum_{m} \sigma_{m}^{-2}\int r_i^2r_j^2 dr_idr_j d\Omega_i d\Omega_j  \iint dk_1dk_2 P(k_1)P(k_2) k_1k_2j_1(k_1|\textbf{r}_i-\textbf{r}_m|)j_1(k_2|\textbf{r}_m-\textbf{r}_j|)\times\\
&\times (\hat{\textbf{r}}_i-\hat{\textbf{r}}_m)\cdot\hat{\textbf{r}}_m(\hat{\textbf{r}}_m-\hat{\textbf{r}}_j)\cdot\hat{\textbf{r}}_m \\
S_4 &= \sum_{i,j} \textbf{D}_{ij}\Delta v_i \Delta v_j \\
&= \frac{\bar{N}a^2H^2f^2b^2}{4\pi^4}\sum_{i,j}\sigma_i^{-2}\sigma_j^{-2}\Delta v_i \Delta v_j \int r^2drd\Omega \iint dk_1dk_2 P(k_1)P(k_2)k_1^2k_2^2(\hat{\textbf{r}_i}-\hat{\textbf{r}})\cdot\hat{\textbf{r}}_i(\hat{\textbf{r}}-\hat{\textbf{r}}_j)\cdot\hat{\textbf{r}}_j\times \\ 
&\times j_1(k_1|\textbf{r}-\textbf{r}_i|)j_1(k_2|\textbf{r}-\textbf{r}_j|) +\sum_{m,i,j} \sigma_i^{-2}\sigma_j^{-2}\sigma_m^{-2}\Delta v_i \Delta v_j\xi_{vv,im}\xi_{vv,mj}\\
S_5 &= 2\sum_{i,j}\textbf{B}_{ij}\Delta v_j \\
&=\frac{\bar{N}aHfb^3}{2\pi^4}\sum_{i,j}\sigma_j^{-2}\Delta v_j\int r^2drd\Omega \iint dk_1dk_2 P(k_1)P(k_2)k_1^2k_2(\hat{\textbf{r}}-\hat{\textbf{r}}_j)\cdot\hat{\textbf{r}}_jj_0(k_1|\textbf{r}-\textbf{r}_i|)j_1(k_2|\textbf{r}-\textbf{r}_j|) \\
&+\frac{\sigma_j^{-2}aHfb}{\pi^2}\sum_{m,i,j}\Delta v_j \sigma_m^{-2} \xi_{vv,mj}\int dk P(k)k j_1(k|\textbf{r}_i-\textbf{r}_m|)(\hat{\textbf{r}}_i-\hat{\textbf{r}}_m)\cdot\hat{\textbf{r}}_m\\
S_6 &=-2\bar{N} \sum_{i\in N_{cell},j\in N_{gal}}\textbf{B}_{ij} \Delta v_j\\
&=\frac{-\bar{N}^2aHfb^3}{2\pi^4}\sum_{j}\sigma_j^{-2}\Delta v_j\int r^2r_i^2drdr_id\Omega d\Omega_i \iint dk_1dk_2 P(k_1)P(k_2)k_1^2k_2(\hat{\textbf{r}}-\hat{\textbf{r}}_j)\cdot\hat{\textbf{r}}_jj_0(k_1|\textbf{r}-\textbf{r}_i|)\times \\
&\times j_1(k_2|\textbf{r}-\textbf{r}_j|)-\frac{\bar{N}aHfb}{\pi^2}\sum_{m,j}\sigma_j^{-2} \Delta v_j \sigma_m^{-2} \xi_{vv,mj}\int r_i^2 dr_id\Omega_i\int dk P(k)k j_1(k|\textbf{r}_i-\textbf{r}_m|)(\hat{\textbf{r}}_i-\hat{\textbf{r}}_m)\cdot\hat{\textbf{r}}_m
\end{align*}\par
The seconder-order contribution from $\textbf{x}^T\textbf{C}^{-1}\textbf{x}$ to the likelihood is thus
\[
A^2 (S_1 +S_2 +S_3 +S_4 +S_5 +S_6)
\]\par
\section{Log det to second order}
We now calculate the second-order contribution from the log det term. The $O(A^2)$ prefactor in $\ln \det(\textbf{I}+A\textbf{X})$ is
\[
-\frac{1}{2}\tr (\textbf{X}^2) = -\frac{1}{2}\tr(\textbf{M}^{-1}\textbf{S}\textbf{M}^{-1}\textbf{M}^{-1}\textbf{S}\textbf{M}^{-1})
\]\par
Since the trace is cyclic, we can permute the matrix multiplication and make it into something we've calculated on page 3:
\begin{align*}
-\frac{1}{2}\tr (\textbf{X}^2)&=-\frac{1}{2}\tr(\textbf{M}^{-1}\textbf{S}\textbf{M}^{-1}\textbf{M}^{-1}\textbf{S}\textbf{M}^{-1})\\
&=-\frac{1}{2}\tr(\textbf{M}^{-1}\textbf{M}^{-1}\textbf{S}\textbf{M}^{-1}\textbf{M}^{-1}\textbf{S}) \\
&= -\frac{1}{2}\tr(\textbf{C}_0^{-1}\textbf{S}\textbf{C}_0^{-1}\textbf{S}) \\
&= -\frac{1}{2}\tr\left\{
\left(
\begin{array}{cc}
\bar{N}^2 \xi_{\delta\delta}^2 + \bar{N}\xi_{\delta v}\bm{\sigma}^{-1} \xi_{v\delta} & \bar{N}\xi_{\delta\delta}\xi_{\delta v}+\xi_{\delta v}\bm{\sigma}^{-1}\xi_{vv} \\
\bar{N}^2\bm{\sigma}^{-1} \xi_{v\delta}\xi_{\delta\delta}+\bar{N}\bm{\sigma}^{-1} \xi_{vv}\bm{\sigma}^{-1} \xi_{v\delta} & \bar{N}\bm{\sigma}^{-1} \xi_{v\delta}\xi_{\delta v} + \bm{\sigma}^{-1} \xi_{vv}\bm{\sigma}^{-1} \xi_{vv}
\end{array}
\right)
\right\}
\end{align*}\par
Expanding the trace, we get (using $D_i$ to label each term)
\begin{align*}
D_1 + D_2 + D_3 + D_4 &\equiv -\frac{\bar{N}^2}{2} \sum_{i,j\in N_{cell}} \xi_{\delta\delta,ij}\xi_{\delta\delta,ji} -\frac{\bar{N}}{2}\sum_{i \in N_{cell},j\in N_{gal}} \sigma_j^{-2}\xi_{\delta v,ij} \xi_{v\delta,ji} \\
&-\frac{\bar{N}}{2} \sum_{i\in N_{gal},j\in N_{cell}} \sigma_i^{-2} \xi_{v\delta,ij}\xi_{\delta v,ji} - \frac{1}{2} \sum_{i,j} \sigma_i^{-2}\sigma_j^{-2} \xi_{vv,ij}\xi_{vv,ji}
\end{align*}\par
Let's calculate each $D$:
\begin{align*}
D_1 &= -\frac{\bar{N}^2 b^4}{8\pi^4} \int r^2r_i^2drdr_id\Omega d\Omega_i d \iint dk_1dk_2 P(k_1)P(k_2) k_1^2k_2^2j_0(k_1|\textbf{r}-\textbf{r}_i|)j_0(k_2|\textbf{r}-\textbf{r}_i|) \\
D_2 &= -\frac{a^2H^2f^2b^2\bar{N}}{8\pi^2} \sum_j \sigma_{j}^{-2}\int r_i^2 dr_i d\Omega_i \iint dk_1dk_2 P(k_1)P(k_2) k_1k_2j_1(k_1|\textbf{r}_i-\textbf{r}_j|)j_1(k_2|\textbf{r}_j-\textbf{r}_i|)\times \\
&\times (\hat{\textbf{r}}_i-\hat{\textbf{r}}_j)\cdot\hat{\textbf{r}}_j(\hat{\textbf{r}}_j-\hat{\textbf{r}}_i)\cdot\hat{\textbf{r}}_j\\
D_3 &= -\frac{\bar{N}a^2H^2f^2b^2}{8\pi^4} \sum_{i}\sigma_i^{-2} \int r^2drd\Omega \iint dk_1dk_2 P(k_1)P(k_2)k_1^2k_2^2(\hat{\textbf{r}_i}-\hat{\textbf{r}})\cdot\hat{\textbf{r}}_i(\hat{\textbf{r}}-\hat{\textbf{r}}_i)\cdot\hat{\textbf{r}}_i\times \\ 
&\times j_1(k_1|\textbf{r}-\textbf{r}_i|)j_1(k_2|\textbf{r}-\textbf{r}_i|)\\
D_4 &=- \frac{1}{2} \sum_{i,j} \sigma_i^{-2}\sigma_j^{-2} \xi_{vv,ij}\xi_{vv,ji}
\end{align*}\par
By similar argument as we did at the end of the previous section, all of the above stay finite in the limit $N_{cell}\to\infty$. In summary, the full likelihood at second order (in $A$, not in $\bar{N}$) is
\[
\boxed{A^2 (S_1 +S_2' +S_4 +S_5 +S_6'+D_2+D_3+D_4)}
\]
where
\begin{align*}
S_2' &=-\frac{a^2H^2f^2b^2\bar{N}}{2\pi^2}\times \\
&\times  \sum_{m,i} \sigma_{m}^{-2}\int r_j^2dr_j d\Omega_j \iint dk_1dk_2 P(k_1)P(k_2) k_1k_2j_1(k_1|\textbf{r}_i-\textbf{r}_m|)j_1(k_2|\textbf{r}_m-\textbf{r}_j|)(\hat{\textbf{r}}_i-\hat{\textbf{r}}_m)\cdot\hat{\textbf{r}}_m \times \\
&\times (\hat{\textbf{r}}_m-\hat{\textbf{r}}_j)\cdot\hat{\textbf{r}}_m \\
S_6' &=-\frac{\bar{N}aHfb}{\pi^2}\sum_{m,j}\sigma_j^{-2} \Delta v_j \sigma_m^{-2} \xi_{vv,mj}\int r_i^2 dr_id\Omega_i\int dk P(k)k j_1(k|\textbf{r}_i-\textbf{r}_m|)(\hat{\textbf{r}}_i-\hat{\textbf{r}}_m)\cdot\hat{\textbf{r}}_m
\end{align*}
after removing the $O(\bar{N}^2)$ terms. Note that $S_3$ and $D_1$ dropped out completely.
\section{Full likelihood to second order}
Adding up both the first and the second order contribution, we have
\[
\mathcal{L} =\boxed{ A^2 (S_1 +S_2'  +S_4 +S_5 +S_6'+D_2+D_3+D_4) +A(s_1+s_2+s_4+s_5+s_6+d_1+d_2)}
\]
with extremum at
\[
\boxed{A^*= -\frac{s_1+s_2+s_4+s_5+s_6+d_1+d_2}{2(S_1 +S_2'  +S_4 +S_5 +S_6'+D_2+D_3+D_4)}}
\]
\end{comment}
\begin{thebibliography}{9}
\bibitem{adam}
C. Adams, C. Blake. 
Improving constraints on the growth rate of structure by modelling the density-velocity cross-correlation in the 6dF Galaxy Survey. arXiv:1706.05205 [astro-ph.CO].
\end{thebibliography}

\end{document}