\documentclass{aastex62}   	% use "amsart" instead of "article" for AMSLaTeX format
\usepackage{graphicx}				% Use pdf, png, jpg, or eps§ with pdflatex; use eps in DVI mode
								% TeX will automatically convert eps --> pdf in pdflatex		
\usepackage{amssymb,amsmath}
\usepackage{natbib,verbatim}
%\usepackage{grffile}
%\usepackage{textcase}
\usepackage{hyperref}
%\usepackage[overload]{textcase}

%SetFonts

%SetFonts

\begin{document}
Consider a galaxy survey that is complete in a volume.  Suppose there are $N_{gal}$ galaxies in the survey.  The volume is divided into $N_{cell} \gg N_{gal}$
equal-volume cells, such that the number of galaxies that occupy cell $i$, $N_i=0$ or $1$.  Suppose that the expected number density is constant over the volume.
The number of expected galaxies per cell and its variance (from Poisson statistics) is $\bar{N} = \frac{N_{\text{gal}}}{N_{\text{cell}}}$.  
Denote the dark matter overdensity correlation function as $\xi^{DM}_{\delta \delta}$. The galaxy overdensity correlation function is
$\xi_{\delta \delta} =  b^2 \xi^{DM}_{\delta \delta}$.  The number-per-cell correlation function 
is $\xi_{NN}  =\bar{N}^2 \xi_{\delta \delta}$.

We are interested in the amplitude of the correlation functions, so we parameterize
 this is $\xi_{NN} =A\bar{N}^2S$.
The expected covariance in galaxy numbers is
\begin{align*}
C &= \bar{N} I + A\bar{N}^2S.
\end{align*}

The likelihood is written as $-2\ln{L} = \ln{\det{C}} + (N-\bar{N})^T C^{-1}(N-\bar{N}) + const$.   Lets Taylor expand about $A=0$ and collect terms that
depend on $A$.  The second term gives
\begin{align*}
A (N-\bar{N})^T S(N-\bar{N})
\end{align*}
The first term gives (check this)
\begin{align*}
A\text{Tr}(\bar{N}S),
\end{align*}
which is goes to zero as we go to infinitely small cells so $\bar{N} \rightarrow 0$.  So we only have to worry about the first term, which when the
cells are very small
\begin{align*}
A (N-\bar{N})^T S(N-\bar{N}) &= A \left( \sum_{i,j \in {galaxy}} S_{ij} -2 \sum_{i \in {galaxy}, j\in {cell}} \bar{N}N_i S_{ij}  +  \sum_{ i,j\in {cell}}  \bar{N}^2 S_{ij}\right).
\end{align*}
The first term is the ``standard'' piece.  The second two pieces are new, which involve integrals over the model correlations.

Lets consider the velocities at the same time.  The covariance matrix looks like
\begin{align*}
C &=
\begin{pmatrix}
\bar{N} I &  0 \\
0&\vec{\sigma}_v^2 I \\
\end{pmatrix}
+A\begin{pmatrix}
\bar{N}^2\xi_{\delta\delta} &  \bar{N}\xi_{\delta v}\\
 \bar{N}\xi_{\delta v}& \xi_{vv} \\
  \end{pmatrix}\\
  & = \left(\sqrt{\bar{N}}I, \vec{\sigma}_v I \right) \left(I + 
  A\begin{pmatrix}
\bar{N}\xi_{\delta\delta} &  \sqrt)\bar{N}) \vec{\sigma}_v \xi_{\delta v}\\
 \bar{N}\xi_{\delta v}& \xi_{vv} \\
  \end{pmatrix}
  \right)\left(\sqrt{\bar{N}}I, \vec{\sigma}_v I \right).
\end{align*}
Again the likelihood is written as $-2\ln{L} = \ln{\det{C}} + (N-\bar{N})^T C^{-1}(N-\bar{N}) + const$.   Lets Taylor expand about $A=0$ and collect terms that
depend on $A$.  

The second term gives
\begin{align*}
A (N-\bar{N})^T S(N-\bar{N})
\end{align*}
The first term gives (check this)
\begin{align*}
A\text{Tr}(\bar{N}S),
\end{align*}
which is goes to zero as we go to infinitely small cells so $\bar{N} \rightarrow 0$.  So we only have to worry about the first term, which when the
cells are very small
\begin{align*}
A (N-\bar{N})^T S(N-\bar{N}) &= A \left( \sum_{i,j \in {galaxy}} S_{ij} -2 \sum_{i \in {galaxy}, j\in {cell}} \bar{N}N_i S_{ij}  +  \sum_{ i,j\in {cell}}  \bar{N}^2 S_{ij}\right).
\end{align*}
The first term is the ``standard'' piece.  The second two pieces are new, which involve integrals over the model correlations.


Taylor expand this around $A=0$.  Then reduce based on cells that have $\sigma_v \rightarrow \infty$.
\begin{comment}


%The overdensity is $\delta_{gg}=-1$ for cells with no galaxies and 
%$\delta_{gg}=\frac{N_{\text{cell}}}{N_{\text{gal}}}-1$ with one galaxy.  The variance in overdensity is $\frac{N_{\text{cell}}}{N_{\text{gal}}}$.
%We approximate the Poisson distribution as being Gaussian (the fact that this is a poor approximation we have to check).
The probability for the data, $\mathbf{x}$ (a vector composed of 0's and 1's) 
is modeled to come from a Gaussian distribution (despite the fact that Poisson statistics better describes the shot noise) with covariance matrix
\begin{align*}
C &= \left(\frac{N_{\text{gal}}}{N_{\text{cell}}}\right)^2\xi_{\delta \delta} + \frac{N_{\text{gal}}}{N_{\text{cell}}}I\\
 &= \left(\frac{N_{\text{gal}}}{N_{\text{cell}}}\right)
 \left(I + \frac{N_{\text{gal}}}{N_{\text{cell}}} \xi_{\delta \delta}\right),
\end{align*}
where $\xi_{\delta \delta}$ is the overdensity correlation. 

The challenge is to approximate the likelihood not using the full matrix but only the small fraction of cells with a galaxy because the cells with zero galaxies drop out,
in the limit of $N_{\text{cell}} \rightarrow \infty$.


For galaxies counts, the likelihood is the probability that there are $N$ galaxies in a certain configuration described by the distances between all of the pairs $\mathbf{r}_{ij}$,
given the model expected galaxy density $n$ and the galaxy two-point correlation function $\xi(r)$, as
$p(\mathbf{r}_{ij};n, \xi(r))$.

Recall the relationship between the joint probability of two galaxies and  the two-point correlation function
$$p(r;n,\xi) = n^2[1+\xi(r)].$$ 

The relationship between the joint probability of three and four galaxies and three- and four-point correlation functions are introduced in \citet{1975ApJ...196....1P} and \citet{1978ApJ...221...19F}.  If the underlying distribution is Normal, the joint probabilities depend only on the two-point function, such that
$$p(r_{12},r_{23},r_{13};n,\xi) = n^3[1+\xi_{12}+\xi_{23}+\xi_{13}]$$
$$p(r_{12},r_{23},r_{13},r_{14},r_{23},r_{24},r_{34};n,\xi) = n^4[1+\xi_{12}+\xi_{23}+\xi_{13}+\xi_{14}+\xi_{23}+\xi_{24}+\xi_{34}+
\xi_{12}\xi_{34}+
\xi_{13}\xi_{24}+
\xi_{14}\xi_{23}].$$
The joint probability of $N$ galaxies is constructed in an analogous manner and is composed 
of the $\xi_{ij}$ of all pairs.

Now consider the case of a simultaneous analysis including both galaxy positions and velocities.
Divide the Universe into $N_{\text{cell}}$ cells whose sizes are small enough that there will only will be one galaxy in each.  The cell centers
are configured angularly (e.g.\ with HEALPIX) such that any given cell is separated from the others by the same set of quantized azimuthal distances.  With
the volume binned based on radius, the same set of distances separate any cell from the others.
Convert the survey containing $N_{\text{gal}}$ galaxies into a galaxy overdensity map, where $\delta_{gg}=-1$ for cells with no galaxies and 
$\delta_{gg}=\frac{N_{\text{cell}}}{N_{\text{gal}}}-1$.  Then we can write a big covariance matrix $C$ for the overdensity and velocity fields. 
The covariance matrix is composed of 4 quadrants.  The density piece is $C_{\delta \delta}[i,j] = A\xi_{\delta \delta}(i,j) + \frac{N_{\text{gal}}}{N_{\text{cell}}}\delta_{ij}$ where $i$ and $j$ are cell indices,
the cross-term piece is $C_{\delta v}[i,j] = A\xi_{\delta v}(i,j)$, and the velocity piece is  $C_{vv}[i,j] =A\xi_{vv}(i,j) +  \sigma^2_v\delta_{ij}$.  For cells
that do not contain a galaxy $\sigma^2_v=\infty$.  Recall the $\xi$ is a function of distance between two points, thus each of these matrices can be ordered
to be ``banded'' matrix.  The eigenvalues of banded matrices are known\footnote{\url{http://lampx.tugraz.at/~hadley/ss2/appendix/bandedmatrices/bandedmatrix.php}}.
The likelihood depends on
$$-2\ln{L} = 
\begin{pmatrix}
\delta_{gg}, v
\end{pmatrix}
\begin{pmatrix}
C_{\delta \delta} & C_{\delta v}\\
C_{\delta v} & C_{vv}
\end{pmatrix}^{-1}
\begin{pmatrix}
\delta_{gg}\\
 v
\end{pmatrix}
+
\ln{\left(\det{\begin{pmatrix}
C_{\delta \delta} & C_{\delta v}\\
C_{\delta v} & C_{vv}
\end{pmatrix}}\right)} + const.
$$
We are not interested in the absolute normalization of the likelihood, so let us consider perturbations around the $A=1$, 
$$-2\ln{L(A)} +  2\ln{L(A=1)}.$$
So we are interested in a Taylor expansion of $L$ about $A=1$ taken to the limit of $N_{\text{cell}} \rightarrow \infty$.

 \begin{thebibliography}{1}
\bibitem[Peebles \& Groth(1975)]{1975ApJ...196....1P} Peebles, P.~J.~E., \& Groth, E.~J.\ 1975, \apj, 196, 1 
\bibitem[Fry \& Peebles(1978)]{1978ApJ...221...19F} Fry, J.~N., \& Peebles, P.~J.~E.\ 1978, \apj, 221, 19 
 \end{thebibliography}
 \end{comment}
\end{document}